\documentclass[10pt,a4paper]{article}
\usepackage[utf8]{inputenc}
\usepackage[russian]{babel}
\usepackage[OT1]{fontenc}
\usepackage{amsmath}
\usepackage{amsfonts}
\usepackage{amssymb}
\usepackage[dvipsnames]{xcolor}
\usepackage{graphicx}
\graphicspath{{Images/}}
\usepackage[left=2cm,right=2cm,top=2cm,bottom=2cm]{geometry}
\usepackage{calc}
\usepackage{wrapfig}
\usepackage{setspace}
\usepackage{indentfirst}
\usepackage{subfigure}
\usepackage{multirow}
\usepackage{amsfonts}
\usepackage{hyperref}
\hypersetup{
    pdfstartview=FitH,  
    linkcolor=black,
    urlcolor=red, 
    colorlinks=true,
    citecolor=blue}
\usepackage{tikz}
\usetikzlibrary{ decorations.markings}

\title{Семинар 18}
\date{\today}

\author{Варламов Антоний Михайлович}

\begin{document}
	\maketitle
	
	\section{Решение краевых задач}
	
	Раньше мы имели дело с решениями задач Коши, которые можно было 
	интерпретировать как решение задачи эволюции. Краевые же задачи скорее 
	относятся к задачам поиска стационарных решений.
	
	\begin{equation}
		\mathbb{L}y = f
	\end{equation}
	
	\begin{equation}
		\mathbb{L}y = \frac{d}{dx}\left(g\left(x\right)\frac{dy}{dx}\right) + 
		q\left(x\right)\frac{dy}{dx} - p\left(x\right)y = f\left(x\right)
	\end{equation}
	
	Для простоты будем рассматривать задачу в одномерном пространстве.
	
	Можно провести некоторую интерпретацию составляющих уравнения:
	
	\begin{equation}
		\begin{cases}
			\frac{d}{dx}\left(g\left(x\right)\frac{dy}{dx}\right) &- \ 
			\text{Диффузионный процесс или процесс теплопереноса}
			\\
			q\left(x\right)\frac{dy}{dx} &- \ \text{трение или перенос вещества 
			скоростью}
			\\
			p\left(x\right)y &- \ \text{сток или источник вещества}
		\end{cases}
	\end{equation}
	
	Более общая постановка требует наличия граничных условий, постановка которых 
	может быть произведена несколькими вариантами:
	
	\begin{enumerate}
		\item Граничные условия Дирихле
		\begin{equation}
			\begin{cases}
				y\left(a\right) = \varphi_{a}
				\\
				y\left(b\right) = \varphi_{b}
			\end{cases}
		\end{equation}
		\item Граничные условия Неймана
		\begin{equation}
			\begin{cases}
				\left.\frac{\partial y}{\partial x}\right|_{x = a} = \chi_{a}
				\\
				\left.\frac{\partial y}{\partial x}\right|_{x = b} = \chi_{b}
			\end{cases}
		\end{equation}
		\item Смешанные условия
		\begin{equation}
			dy\left(a\right) + \left.\beta\frac{\partial y}{\partial x}\right|
			_{x = a} = \Phi_{a}
		\end{equation}
	\end{enumerate}
	
	Рассмотрим пример:
	
	\begin{equation}
		\begin{cases}
			\frac{d^{2}y}{dx^{2}} = f\left(x\right), x\in \left[a, b\right];
			\\
			y\left(a\right) = \varphi_{a};
			\\
			y\left(b\right) = \varphi_{b};
		\end{cases}
	\end{equation}
	
	Для построения численного решения необходимо:
	
	\begin{enumerate}
		\item Сетка
		
		\begin{equation}
			h = \frac{b - a}{N}
		\end{equation}
		\begin{equation}
			x_{i} = a + ih, i \in \left[0, N\right]
		\end{equation}
		\item Приближение второй производной
		
		\begin{equation}
			\frac{d^{2}}{dx^{2}} \approx \frac{y_{i + 1} - 2y_{i} + y_{i - 1}}
			{h^{2}}
		\end{equation}
		
		\begin{equation}
			\begin{cases}
				\frac{y_{i + 1} - 2y_{i} + y_{i - 1}}{h^{2}} = f_{i}, &i \in 
				\left[2, N - 2\right];
				\\
				\frac{y_{2} - 2y_{1} + \varphi_{a}}{h^{2}} = f_{1}, &i = 1;
				\\
				\frac{\varphi_{b} - 2y_{N - 1} + y_{N - 2}}{h^{2}} = f_{N - 1},
				&i = N - 1;
			\end{cases}
		\end{equation}
		
		Данные выражения можно интерпретировать как систему уравнений, для 
		которой существует матричное представление.
		
		\item Матричное представление
		
		\begin{equation}
			\begin{cases}
				\vec{y} = \left(y_{1},y_{2}, \ldots, y_{N - 1}\right)^{T}
				\\
				\vec{f} = \left(f_{1} - \frac{\varphi_{a}}{h^{2}},f_{2}, \ldots, 
				f_{N - 2}, f_{N - 1} - \frac{\varphi_{b}}{h^{2}}\right)^{T}
				\\
				A\cdot\vec{y} = \vec{f}
			\end{cases}
		\end{equation}
		
		Где матрица $А$ имеет вид:
		
		\begin{equation}
			A = \frac{1}{h^{2}}\begin{pmatrix}
				-2 & 1 & 0 & \ldots & \ldots &0 \\
				1 & -2 & 1 & \ldots & \ldots & 0 \\
				0& \ddots & \ddots &\ddots & \ddots & 0\\
				\vdots & \ddots & \ddots &\ddots & \ddots & \vdots\\
				\vdots & \ddots & \ddots &\ddots & \ddots & 1\\
				0 & \ldots & \ldots & \ldots & 1 & -2 
			\end{pmatrix}, \left(Ax, x\right) \leqslant 0
		\end{equation}
		
		\item Обращение матрицы и получение решения:
		
		\begin{equation}
			\vec{y} = A^{-1}\vec{f}
		\end{equation}
	\end{enumerate}
	
	Рассмотрим чуть подробнее описанный метод. Для этого рассмотрим выражения:
	
	\begin{equation}
		\begin{cases}
			\frac{d^{2}u}{dx^{2}} = f\left(x\right)
			\\
			u\left(a\right) = \varphi_{a}
			\\
			u\left(b\right) = \varphi_{b}
		\end{cases}
	\end{equation}
	
	\begin{align}
		r_{i} &= \frac{u_{i + 1} - 2u_{i} + u_{i - 1}}{h^{2}} - f_{i} = \\
		&= \frac{u_{i} + hu_{i}' + \frac{h^{2}}{2}u_{i}'' + \frac{h^{3}}{6}
		u_{i}''' + \frac{h^{4}}{24}u_{i}^{\left(IV\right)} - 2u_{i} + u_{i}- 
		hu_{i}' + \frac{h^{2}}{2}u_{i}'' - \frac{h^{3}}{6}u_{i}''' + 
		\frac{h^{4}}{24}u_{i}^{\left(IV\right)}}{h^{2}} - f_{i} =\\
		&= u_{i}'' - f_{i} + \frac{h^{2}}{24}\left(u_{i}^{\left(IV\right)}
		\left(\theta_{1}\right) + u_{i}^{\left(IV\right)}\left(\theta_{2}\right)
		\right)
	\end{align}
	
	\begin{equation}
		\max \left|r_{i}\right| = \parallel r_{i}\parallel_{\infty} \leqslant
		\frac{h^{2}}{24}\cdot 2M_{4}
	\end{equation}
	
	Рассмотрим теперь другие граничные условия:
	
	\begin{equation}
		\left.\frac{\partial u}{\partial x}\right|_{x = a} \approx \frac{u
		\left(a + h\right) - u\left(a\right)}{h}
	\end{equation}
	
	В таком случае:
	
	\begin{align}
		i &= 0: \frac{y_{1} - y_{0}}{h} = \chi_{a}
		\\
		i &= 1: \frac{y_{2} - 2y_{1} + y_{0}}{h^{2}} = f_{1}
		\\
		&\vdots		
	\end{align}
	
	Матрица приобретет другой вид:
	
	\begin{equation}
			A' = \frac{1}{h^{2}}\begin{pmatrix}
				-1 & 1 & 0 & \ldots & \ldots &0 \\
				1 & -2 & 1 & \ldots & \ldots & 0 \\
				0& \ddots & \ddots &\ddots & \ddots & 0\\
				\vdots & \ddots & \ddots &\ddots & \ddots & \vdots\\
				\vdots & \ddots & \ddots &\ddots & \ddots & 1\\
				0 & \ldots & \ldots & \ldots & 1 & -2 
			\end{pmatrix}
	\end{equation}
	
	Что приводит к потери порядка аппроксимации. Методов борьбы с этим 
	существует несколько, вот одни из них:

	\begin{enumerate}
		\item Взять лучшее приближение производной в определении граничных 
		условий:
	
	\begin{equation}
		\left.\frac{\partial u}{\partial x}\right|_{x = a} \approx 
		\frac{1}{2h}\left(-3y_{0} + 4y_{1} - y_{2}\right)
	\end{equation}
	
	\begin{equation}
			A'' = \frac{1}{h^{2}}\begin{pmatrix}
				-3h & 2h & -h & \ldots & \ldots &0 \\
				1 & -2 & 1 & \ldots & \ldots & 0 \\
				0& \ddots & \ddots &\ddots & \ddots & 0\\
				\vdots & \ddots & \ddots &\ddots & \ddots & \vdots\\
				\vdots & \ddots & \ddots &\ddots & \ddots & 1\\
				0 & \ldots & \ldots & \ldots & 1 & -2 
			\end{pmatrix}
	\end{equation}
	
	\item \textit{Метод фиктивной точки}. Идея данного метода заключается в 
	следующем:
	
	\begin{equation}
		i = 0: \frac{y_{1} - 2y_{0} + y_{-1}}{h^{2}} = f_{0}
	\end{equation}
	
	С другой стороны, 
	
	\begin{equation}
		\frac{y_{1} - y_{-1}}{2h} = \chi_{a} \rightarrow y_{-1} = -2h\chi_{a} + 
		y_{1}
	\end{equation}
	
	\begin{equation}
		\frac{y_{1} - 2y_{0} + y_{1} - 2h\chi_{a}}{h^{2}} = f_{0}
	\end{equation}
	
	Для такой схемы матрица будет иметь вид:
	\begin{equation}
			A''' = \frac{1}{h^{2}}\begin{pmatrix}
				-2 & 2 & 0 & \ldots & \ldots &0 \\
				1 & -2 & 1 & \ldots & \ldots & 0 \\
				0& \ddots & \ddots &\ddots & \ddots & 0\\
				\vdots & \ddots & \ddots &\ddots & \ddots & \vdots\\
				\vdots & \ddots & \ddots &\ddots & \ddots & 1\\
				0 & \ldots & \ldots & \ldots & 1 & -2 
			\end{pmatrix}
	\end{equation}
	
	\end{enumerate}
	
	
	\textit{Метод пристрелки}. Рассмотрим уравнение вида:
	
	\begin{equation}
		\begin{cases}
			\frac{d^{2}u}{dx^{2}} - u^{2} = f\left(x\right)
			\\
			u\left(a\right) = \varphi_{a}
			\\
			u\left(b\right) = \varphi_{b}
		\end{cases}
	\end{equation}
	
	\begin{equation}
		\begin{cases}
			\frac{d^{2}u}{dx^{2}} - u^{2} = f\left(x\right)
			\\
			u\left(a\right) = \varphi_{a}
			\\
			u\left(b\right) = \varphi_{b}
		\end{cases} \xrightarrow{Перейдем к задаче Коши}
		\begin{cases}
			\frac{du}{dx} = z
			\\
			\frac{dz}{dx} - u^{2} = f\left(x\right)
			\\
			u\left(a\right) = \varphi_{a}
			\\
			z\left(a\right) = \alpha
		\end{cases}
	\end{equation}
	
	После перехода получаем $u\left(b\right)$, которое в общем случае 
	
	\begin{equation}
		u\left(b\right) \neq \varphi_{b}
	\end{equation}
	
	Однако, не стоит забывать, что на самом деле нами был введен дополнительный 
	параметр, относительно которого необходимо решить уравнение:
	\begin{equation}
		u\left(b, \alpha\right) - \varphi_{b} = 0;
	\end{equation}
	
\end{document}