\documentclass[12pt]{article}

%\usepackage{showframe}
\usepackage{graphicx}
\usepackage{amsmath}
\usepackage[margin=1in]{geometry}
\usepackage{cmap} 			% для поиска по русским символам в полученном PDF
\usepackage[T2A]{fontenc} 
\usepackage[utf8]{inputenc}
\usepackage[english,russian]{babel} % лучше оба языка --- для единообразия!
\usepackage{pscyr}
\usepackage{amsthm}
\usepackage{mathtools}
\usepackage{subcaption}
\usepackage{varwidth}
\usepackage{parskip}

\newcommand{\pt}{\partial t}
\newcommand{\px}{\partial x}
\newcommand{\p}{\partial}
\newcommand{\dt}{\Delta t}
\newcommand{\dx}{\Delta x}

\newcommand{\ff}[2]{{\psi_{#1}^{#2}}}

\newtheorem*{theorem}{Теорема}
\newtheorem*{statement}{Утверждение}
\newtheorem*{conseq}{Следствие}

\begin{document}


\section{Тайны вычислительной математики. Общая теория сходимости уравнений в частных производных}

Переходим к решению уравнений и систем уравнений в частных производных. \\
Некоторые примеры таких уравнений в одномерном случае:
\begin{itemize}
	\item уравнение теплопроводности (диффузии)
	\begin{align}
	\frac{\p \psi}{\pt} =\kappa \frac{\p^2 \psi}{\px^2}
	\end{align}
	\item уравнение переноса (адвекции)
	\begin{align}
	\frac{\p \psi}{\pt} + a \frac{\p \psi}{\px} = 0
	\end{align}
	\item волновое уравнение
	\begin{align}
	\frac{\p^2 \psi}{\pt^2} = \frac{1}{c^2} \frac{\p^2 \psi}{\px^2}
	\end{align}
	\item Линейные уравнения мелкой воды
	\begin{align}
	\frac{\p u}{\pt} = -g \frac{\p h}{\px} \\
	\frac{\p h}{\pt} = -\bar{h} \frac{\p u}{\px}
	\end{align}
\end{itemize}
Зададимся вопросом построения алгоритма для численного решения уравнений в частных производных. Ранее мы занимались решением задач (задача Коши для ОДУ, краевая задача) с одной размерностью -- время или пространство, теперь же у нас присутствует обе размерности. То есть, грубо говоря, мы одновременно решаем и задачу Коши, и краевую задачу.\\
\\
\subsection{Пример. Численное решение уравнения теплопроводности}
\subsubsection{Постановка задачи}
Начнем с простого примера -- построим алгоритм численного решения следующей задачи:
\begin{align}\label{eq_dif}
&\frac{\p \psi}{\pt} = \frac{\p^2 \psi}{\px^2}, x\in[0,1],\ t\in [0,T]. \\
&\psi(x,0) = \varphi(x), \\
&\psi(0,t) = a, \psi(1,t) = b \label{boundary}
\end{align}
\subsubsection{Дикретизация уравнения}
Для численного решения нам требуется ввести сетку по пространству и по времени. Для простоты будем пользоваться равномерными сетками:
\begin{align}
	&\dt = \frac{T}{N_t}, t_n = \dt(n-1), n=0..N_t. \\
	&\dx = \frac{1}{N_x}, x_i = \dx(i-1), i=0..N_x.
\end{align}
Здесь $N_t, N_x$ -- количество узлов сетки по времени и по пространству, $\dt, \dx$ -- шаги сетки, $t_n, x_i$ -- координаты узлов сетки. \\
Значения сеточных функции в узлах сетки будем обозначать $\psi_i^n$, где нижний индекс -- номер узла сетки по пространству, верхний индекс -- номер узла сетки по времени.


Запишем теперь простейшую дискретизацию уравнения (\ref{eq_dif}), для этого для аппроксимации частной производной по времени воспользуемся формулой направленной разности первого порядка, а для аппроксимации второй производной по пространству стандартной формулой второго порядка аппроксимации. Получаем схему:
\begin{align}\label{expl}
\frac{\ff{i}{n+1} - \ff{i}{n}}{\dt} = \frac{\ff{i+1}{n}-2\ff{i}{n}+\ff{i-1}{n}}{\dx^2}, n=1..N_t, i=2..N_x-2.
\end{align}
Для приграничных узлов сетки $x_1, x_{N_x-1}$ нужно будет учесть граничные условия (в данном случае условия типа Дирихле) (\ref{boundary}) и подставить $\psi_0^n=a, \psi_{N_x}^n=b$. В целом способы аппроксимации граничных условий для уравнений в частных производных не сильно отличаются от способов, которые были разобраны в теме про краевые задачи.\\
Таким образом мы только что построили простейшую \textbf{явную} схему. Аналогичным образом мы могли построить неявную схему:
\begin{align}\label{impl}
\frac{\ff{i}{n+1} - \ff{i}{n}}{\dt} = \frac{\ff{i+1}{n+1}-2\ff{i}{n+1}+\ff{i-1}{n+1}}{\dx^2}, n=1..N_t, i=2..N_x-2.
\end{align}
При использовании такой схемы на каждом шаге по времени придется решать систему линейных уравнений с трехдиагональной матрицей.\\
В целом можно рассматривать отдельно аппроксимацию по времени и пространству, то есть сначала перейти к системе обыкновенных дифференциальных уравнений относительно неизвестных $\psi_i(t)$:
\begin{align}\label{sd}
\frac{d \psi_i(t)}{d t} = \frac{\psi_{i+1}(t)-2\psi_i(t)+\psi_{i-1}(t)}{\dx^2},
\end{align}
после чего применить любой известный нам алгоритм численного интегрирования ОДУ (методы Рунге-Кутты, методы Адамса, ...). В частности, применение явного или неявного метода Эйлера для системы (\ref{sd}) приводит к схемам (\ref{expl}), (\ref{impl}). Такой подход к решению уравнений в частных производных принято называть \textbf{методом прямых} (method of lines).\\
\subsubsection{Исследование аппроксимации}
Вернемся к явной схеме:
 \begin{align}\label{expl2}
 \frac{\ff{i}{n+1} - \ff{i}{n}}{\dt} = \frac{\ff{i+1}{n}-2\ff{i}{n}+\ff{i-1}{n}}{\dx^2}.
 \end{align}
Теперь нам надо понять насколько решение, полученное при помощи такой схемы, будет похоже на настоящее аналитическое решения исходного уравнения. Для этого нам нужно показать, что есть сходимость разностного решения к аналитическому (определение сходимости можно посмотреть в задачнике, оно слабо отличается от определения, которое мы использовали для ОДУ). Как и раньше доказывать сходимости напрямую сложно, поэтому будем доказывать аппроксимацию и устойчивость.

Для исследования аппроксимации нужно показать, что норма вектора невязки стремится к нулю при стремлении к нулю $\dx$ и $\dt$. Выпишем выражение для компоненты вектора невязки $r_i^n$:
\begin{align}\label{res}
	r_i^n = \frac{[\psi]_{i}^{n+1} - [\psi]_{i}^{n}}{\dt} - \frac{[\psi]_{i+1}^{n}-2[\psi]_{i}^{n}+[\psi]_{i-1}^{n}}{\dx^2}.
\end{align}
Здесь $[\psi]_{i}^{n} \equiv \psi(t_n,x_i)$ -- проекция точного решения уравнения (\ref{eq_dif}) на сетку. Воспользуемся разложением в ряд Тейлора:
\begin{align}
&[\psi]_{i}^{n+1} = [\psi]_{i}^{n} + \dt [\psi'_t]_{i}^{n} + \underbracket[0.7pt]{\frac{\dt^2}{2}\psi''_{tt}(\theta_1)}_{\substack{\text{Остаточный член} \\ \text{в форме Лагранжа}}} \\
&[\psi]_{i \pm 1}^{n} = [\psi]_{i}^{n} \pm \dx [\psi'_x]_{i}^{n} + \frac{\dx^2}{2}[\psi''_{xx}]_{i}^{n} \pm \frac{\dx^3}{6}[\psi'''_{xxx}]_i^n + \frac{\dx^4}{24}\psi''''_{xxxx}(\theta_{2/3}).
\end{align} 
Подставляя все в (\ref{res}) получаем:
\begin{align}\label{res}
r_i^n = [\psi'_t]_{i}^{n}+\frac{\dt}{2}\psi''_{tt}(\theta_1) - [\psi''_{xx}]_{i}^{n} - \frac{\dx^2}{24}(\psi''''_{xxxx}(\theta_{2})+\psi''''_{xxxx}(\theta_{3})).
\end{align}
$[\psi_t]_{i}^{n}$ -- проекция точного решения на сетку $\Rightarrow$ $[\psi'_t]_{i}^{n}=[\psi''_{xx}]_{i}^{n}$.
\begin{align}\label{res}
|r_i^n| =\left| \frac{\dt}{2}\psi''_{tt}(\theta_1) - \frac{\dx^2}{24}(\psi''''_{xxxx}(\theta_{2})+\psi''''_{xxxx}(\theta_{3}))\right| \le \\
\frac{\dt}{2} \max_{t\in [0,T]} |\psi''_{tt}(t)| + \frac{\dx^2}{12}\max_{x\in [0,1]}|\psi''''_{xxxx}(x)|\equiv C_1 \dt + C_2 \dx^2.
\end{align}
То есть схема аппроксимирует исходное уравнение с первым порядком по времени и вторым порядком по пространству.\\
\subsubsection{Исследование устойчивости}
Перейдем теперь к исследованию устойчивости разностной схемы (\ref{expl2}). Проводить исследование устойчивости по определению достаточно сложно, поэтому воспользуемся так называемым \textbf{спектральным признаком} устойчивости (признак Неймана). Для этого подставим в разностную схему выделенную Фурье-гармонику $\psi_k^n=\lambda^n e^{i\alpha k}$ (далее индекс $i$ заменим на $k$, чтобы не путать с мнимой единицей) и потребуем, чтобы такие решения были не возрастающими, то есть чтобы $|\lambda| \le 1$. Получаем следующее выражение:
\begin{align}
\frac{\lambda^{n+1} e^{i\alpha k}-\lambda^{n} e^{i\alpha k}}{\dt} = \frac{\lambda^{n} e^{i\alpha (k+1)}-2\lambda^{n} e^{i\alpha k}+\lambda^{n} e^{i\alpha (k-1)}}{\dx^2}
\end{align}
Можем сократить обе части на $\lambda^n$ и на $e^{i\alpha k}$:
\begin{align}
\frac{\lambda - 1}{\dt} = \frac{e^{i\alpha} - 2 +e^{-i\alpha}}{\dx^2}.
\end{align}
Отсюда можем получить выражение для $\lambda$:
\begin{align}
\lambda =1 +\frac{2\dt}{\dx^2} (\cos{\alpha}-1) = 1 - \frac{4\dt}{\dx^2} \sin^2\frac{\alpha}{2}.
\end{align}
Условие устойчивости:
\begin{align}
&|\lambda|\le 1 \Rightarrow  \left|1 - \frac{4\dt}{\dx^2} \sin^2\frac{\alpha}{2}\right| \le 1 \Rightarrow \\
&\Rightarrow 0 \le \frac{4\dt}{\dx^2} \sin^2\frac{\alpha}{2} \le 2.
\end{align}
Учитывая, что $\sin^2\frac{\alpha}{2}$ меньше или равен единицы, получаем условие на $\dt$ и $\dx$ при которых $|\lambda|\le1$:
\begin{align}
0\le\frac{\dt}{\dx^2}\le\frac{1}{2}.
\end{align}
\subsection{Метод неопределенных коэффициентов}
Рассмотрим еще один подход к построению аппроксимаций для уравнений в частных производных -- метод неопределенных коэффициентов. Идея метода состоит в том, чтобы сначала выбрать шаблон аппроксимации, то есть набор узлов сетки, которые вы будете использовать для построения схемы, и записать вашу схему как сумму значений сеточной функции в этих узлах с неопределенными коэффициентами. Давайте разберем этот подход на конкретном примере.
\subsubsection{Пример. Численное решение уравнения переноса}
Требуется построить численную схему для решения уравнения переноса:
\begin{align}
	\frac{\p \psi}{\pt} + s \frac{\p \psi}{\px} = 0
\end{align}
методом неопределенных коэффициентов. Качестве шаблона аппроксимации будем использовать узлы сетки $(x_k,t_n)$, $(x_k,t_{n+1})$,$(x_{k-1},t_{n+1})$,$(x_{k+1},t_{n+1})$. Иногда шаблоны аппроксимации изображают схематично в виде линий, соединящих узлы сетки в пространстве $(x,t)$.\\
Запишем общий вид нашей схемы:
\begin{align}
a\psi_{k}^{n}+b\psi_{k}^{n+1}+c\psi_{k+1}^{n+1}+d\psi_{k-1}^{n+1}=0.
\end{align} 
Здесь $a,b,c,d$ -- неопределенные коэффициенты, которые нам надо определить из условий на порядок аппроксимации схемы. Для этого подставим вместо сеточной функции проекцию точного решения на сетку и разложим все в ряд Тейлора, расскладывать будет удобно относительно точки $(x_k,t_{n+1})$.
\begin{align}\label{undef}
a[\psi]_{k}^{n}+b[\psi]_{k}^{n+1}+c[\psi]_{k+1}^{n+1}+d[\psi]_{k-1}^{n+1}=0.
\end{align}
\begin{align}
&[\psi]_{k\pm 1}^{n+1}=[\psi]_{k}^{n+1} \pm \frac{\dx}{1}[\psi'_x]_{k}^{n+1}+\frac{\dx^2}{2}[\psi''_{xx}]_{k}^{n+1} \pm \frac{\dx^3}{6}[\psi'''_{xxx}]+\frac{\dx^4}{24}[\psi''''_{xxxx}]_{k}^{n+1}\\
&[\psi]_{k}^{n}=[\psi]_{k}^{n+1} - \frac{\dt}{1}[\psi'_t]_{k}^{n+1}+\frac{\dt^2}{2}[\psi''_{tt}]_{k}^{n+1} - \frac{\dt^3}{6}[\psi'''_{ttt}]+\frac{\dt^4}{24}[\psi''''_{tttt}]_{k}^{n+1}
\end{align}
Подставив эти разложения в (\ref{undef}) потребуем сначала, чтобы коэффициент перед $[\psi]^{n+1}_k$ был равен нулю (такого члена нет в исходном уравнении), коэффициент перед $[\psi'_t]^{n+1}_k$ был равен $1$, коэффициент перед $[\psi'_x]^{n+1}_k$ был равен $s$. Получаем три условия на коэффициенты:
\begin{align}
&a+b+c+d=0,\\
&-a\dt=1, \\
&(c-d)\dx=s.
\end{align}
Далее для возможности получить схему максимального порядка аппроксимации выразим производные по времени через производные по пространству, воспользовавшись дифференциальными следствиями исходного уравнения:
\begin{align}
	&\psi'_t+s\psi'_x = 0 \Rightarrow
\end{align}
	\begin{align}
	&\psi''_{tt}+s\psi''_{tx}=\psi''_{tt}+s(\psi'_{t})'_x=\psi''_{tt}+s(-s\psi'_x)'_x=0 \Rightarrow
\end{align}
\begin{align}
&\psi''_{tt}=s^2\psi''_{xx}
\end{align}
Теперь можем потребовать, чтобы коэффициент перед $[\psi''_{xx}]^{n+1}_k$ был равен нулю:
\begin{align}
	\frac{\dx^2}{2}(c+d)+as^2\frac{\dt^2}{2}=0.
\end{align}
Таким образом мы получили 4 уравнения для четырех неизвестных. Решение этой системы:
\begin{align}
	a=-\frac{1}{\dt}, b=\frac{1}{\dt}-\frac{s^2\dt}{\dx^2}, c=\frac{1}{2\dx}+\frac{s^2\dt}{2\dx^2}, d=-\frac{1}{2\dx}+\frac{s^2\dt}{2\dx^2}. 
\end{align}
Если подставить значения коэффициентов в (\ref{undef}), получим схему:
\begin{align}
	\frac{\psi_{k}^{n+1}-\psi_{k}^{n}}{\dt} + s \frac{\psi_{k+1}^{n+1}-\psi_{k-1}^{n+1}}{2\dx}+\frac{s^2\dt}{2\dx^2}(\psi_{k+1}^{n+1}-2\psi_{k}^{n+1}+\psi_{k-1}^{n+1})=0.
\end{align}
Такая схема для уравнения переноса называется неявной схемой Лакса-Вендроффа. По построению эта схема второго порядка аппроксимации по времени и по пространству. 

\end{document}



