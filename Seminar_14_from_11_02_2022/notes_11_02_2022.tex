\documentclass[10pt,a4paper]{article}
\usepackage[utf8]{inputenc}
\usepackage[russian]{babel}
\usepackage[OT1]{fontenc}
\usepackage{amsmath}
\usepackage{amsfonts}
\usepackage{amssymb}
\usepackage[dvipsnames]{xcolor}
\usepackage{graphicx}
\graphicspath{{Images/}}
\usepackage[left=2cm,right=2cm,top=2cm,bottom=2cm]{geometry}
\usepackage{calc}
\usepackage{wrapfig}
\usepackage{setspace}
\usepackage{indentfirst}
\usepackage{subfigure}
\usepackage{multirow}
\usepackage{amsfonts}
\usepackage{hyperref}
\hypersetup{
    pdfstartview=FitH,  
    linkcolor=black,
    urlcolor=red, 
    colorlinks=true,
    citecolor=blue}
\usepackage{tikz}
\usetikzlibrary{ decorations.markings}

\title{Семинар 14}
\date{\today}
\author{Варламов Антоний Михайлович}

\begin{document}
	\maketitle

	% 7 тем. Для засчитывания минимально 2 теор + практ (если нет практ, то 3 
	% теор).
	% 2 Контрольные работы (АКР и ПКР)
	% Летучки работают так же
	% Необходимо выполнить и сдать проект -- максимум 20 баллов за проект

	
	\section{Решение ОДУ}
	
	Рассмотрим задачу:
	
	\begin{equation}
		\begin{cases}
			\frac{d\vec{u}}{dt} = \vec{f}\left(u, t\right)
			\\
			\vec{u}\left(0\right) = \vec{u}_{0}
		\end{cases}
	\end{equation}
	
	Построим переход от дифференциальной к разностной формулировке. Для этого, 
	в частности можно рассмотреть схему:
	
	\begin{equation}
		\frac{du}{dt} \approx \frac{u\left(t + \Delta t\right) - u\left(t
		\right)}{\Delta t}
	\end{equation}
	
	Рассмотрим промежуток $\left[0; T\right]$. Введем стандартные обозначения:
	
	\begin{equation}
 		\Delta t = \frac{T}{N_{T}},  \ \ \ t_{i} = i\cdot\Delta t, \ i \in 
 		\left[0, N\right]
	\end{equation}
	
	Для обозначения численного решения будем использовать $y_{i}$. Построим 
	простейшую разностную схему:
	
	\begin{equation}
		\begin{cases}
			\frac{y_{i + 1} - y_{1}}{\Delta t} = f\left(y_{i}, t_{i}\right), \ \ 
			i \in \left[0, N\right];
			\\
			y_{0} = u_{0};
		\end{cases}
	\end{equation}
	
	Рассмотрим уравнение, где $f\left(u, t\right) = \lambda u$
	
	%Вставить картинки для решения уравнения!!!!!!!!!!!!!!!!!!!!!!!!!!!!!!!!!!
	
	% 15 и 30 узлов, от 0 до 3
	% 15 узлов, от 0 до 5, от 0 до 7, от 0 до 9, от 0 до 15
	
	Рассмотрим другое представление данной задачи:
	
	\begin{equation}
		\frac{y_{i + 1} - y_{i}}{\Delta t} = \frac{f\left(y_{i + 1}, t_{i + 1}
		\right) + f\left(y_{i}, t_{i}\right)}{2}
	\end{equation}
	
	Выразим $y_{i + 1}$:
	
	\begin{equation}
		y_{i + 1} = 
		y_{i} + \frac{\Delta t}{2}\left(f\left(y_{i + 1}, t_{i + 1}\right) + 
		f\left(y_{i}, t_{i}\right)\right)
	\end{equation}
	
	\begin{equation}
		f\left(u, t\right) = \lambda u \Rightarrow y_{i + 1} = 
		\frac{\left(1 + \frac{\Delta t\lambda}{2}\right)}{\left(1 - 
		\frac{\Delta t\lambda}{2}\right)}y_{i}
	\end{equation}
	
	%вставить картинки для новой схемы
	
	Рассмотрим всю ту же задачу:
	
	\begin{equation}
		\begin{cases}
			\frac{d\vec{u}}{dt} = \vec{f}\left(u, t\right)
			\\
			\vec{u}\left(0\right) = \vec{u}_{0}
		\end{cases}
	\end{equation}
	
	Перейдем к операторному виду данного уравнения:
	
	\begin{equation}
		\hat{\mathbb{L}} = f
	\end{equation}
	
	Где $\hat{\mathbb{L}}: U \rightarrow V$, $u \in D\left(\mathbb{L}\right), 
	u \in U, f \in R\left(\mathbb{L}\right)$
	
	Операторное уравнение также может быть сведено к разностной схеме:
	
	\begin{equation}
		\mathbb{L}_{n}y_{n} = f_{n},  \ \ \ y_{n} \in \mathbb{R}^{N_{T}}, 
		f_{n}\in \mathbb{R}^{N_{T}}
 	\end{equation}
 	
 	Определим оператор проекции на сетку:
 	
 	\begin{equation}
 		\mathbb{P}_{h}: U \rightarrow \mathbb{R}^{N_{T}}
 	\end{equation}
 	
 	\textit{\textbf{Определение:} Решение разносного уравнения сходится к 
 	решению операторного уравнения, если }
 	
 	\begin{equation}
 		\parallel \mathbb{P}_{h} u - y_{n}\parallel \leqslant C\cdot\Delta t^{p}
 	\end{equation}
 	
 	\textit{где $p$ -- порядок сходимости}
 	
 	\textit{\textbf{Теорема} Если имеется аппроксимация, а также используемая 
 	схема устойчива, то имеется сходимость}%Написать хорошую формулировку
 	
 	\textit{\textbf{ОпределениеЪ} Разностная схема * аппроксимирует **, если:}
 	
 	\begin{equation}
 		\parallel \mathbb{L}_{h}\left(\mathbb{P}_{h}u\right) - \mathbb{L}y_{h}
 		\parallel \xrightarrow{\Delta t \to 0} 0
 	\end{equation}
 	
 	\begin{equation}
 		\parallel \mathbb{L}_{h}\left(\mathbb{P}_{h}u\right) - \mathbb{L}y_{h}
 		\parallel \leqslant C\Delta t^{p}
 	\end{equation}
 	
 	Рассмотрим выражение:
 	
 	\begin{equation}
 		\frac{y_{i + 1} - y_{i}}{\Delta t} = f\left(t_{i}, y_{i}\right)
 	\end{equation}
 	
 	Определим 
 	
 	\begin{equation}
 		r_{i} = \frac{u_{i + 1} - u_{i}}{\Delta t} - f\left(t_{i}, u_{i}\right)
 	\end{equation}
	
	
	
\end{document}