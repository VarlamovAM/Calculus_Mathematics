\documentclass[10pt,a4paper]{article}
\usepackage[utf8]{inputenc}
\usepackage[russian]{babel}
\usepackage[OT1]{fontenc}
\usepackage{amsmath}
\usepackage{amsfonts}
\usepackage{amssymb}
\usepackage[dvipsnames]{xcolor}
\usepackage{graphicx}
\graphicspath{{Images/}}
\usepackage[left=2cm,right=2cm,top=2cm,bottom=2cm]{geometry}
\usepackage{calc}
\usepackage{wrapfig}
\usepackage{setspace}
\usepackage{indentfirst}
\usepackage{subfigure}
\usepackage{multirow}
\usepackage{amsfonts}
\usepackage{hyperref}
\hypersetup{
    pdfstartview=FitH,  
    linkcolor=black,
    urlcolor=red, 
    colorlinks=true,
    citecolor=blue}
\usepackage{tikz}
\usetikzlibrary{ decorations.markings}

\title{Семинар 16}
\date{\today}

\author{Варламов Антоний Михайлович}

\begin{document}
		\maketitle
	
	\section{Неявные методы Рунге-Кутты}
	
	
	В качестве примера неявных методов Рунге-Кутты рассмотрим семейство 
	двухстадийных методов общего вида с таблицей Бутчера:
	
	\begin{table}[h!]
\centering
\begin{tabular}{l|ll}
\multicolumn{1}{l|}{$c_{1}$}             & $\beta_{11}$             & $
\beta_{12}$ \\
\multicolumn{1}{l|}{$c_{2}$} & $\beta_{21}$ & $\beta_{22}$ \\ \hline
                                   & $b_{1}$           & $b_{2}$
\end{tabular}
\caption{Таблица Бутчера для общего вида двухстадийных неявных методов Рунге-
Кутты.}
\label{tab:Butcher_tab_example_1}
\end{table}

Вычислительные формулы, соответствующие данной таблице Бутчера:
	
	\begin{equation}
		\begin{cases}
			T_{1} = t_{n} + c_{1}\Delta t
			\\
			T_{2} = t_{n} + c_{2}\Delta t
		\end{cases}
	\end{equation}
	
	\begin{equation}
		\begin{cases}
			Y_{1} = y_{n} + \Delta t\left(\beta_{11}\cdot f\left(T_{1},Y_{1}
			\right) + \beta_{12}\cdot f\left(T_{2}, Y_{2}\right)\right)
			\\
			Y_{2} = y_{n} + \Delta t\left(\beta_{21}\cdot f\left(T_{1},Y_{1}
			\right) + \beta_{22}\cdot f\left(T_{2}, Y_{2}\right)\right)			
		\end{cases}
	\end{equation}
	
	\begin{equation}
		y_{n + 1} = y_{n} + \Delta t\left(b_{1}\cdot f\left(T_{1}, Y_{1}\right)
		+ b_{2}\cdot f\left(T_{2}, Y_{2}\right)\right)
	\end{equation}
	
Здесь, в отличие от явных методов Рунге-Кутты, на каждом шаге по времени 
необходимо решать систему нелинейных уравнений относительно величин $Y_1$, 
$Y_2$. Существенное преимущество неявных методов -- их хорошая устойчивость.

\section{Многошаговые методы}

Перейдем к рассмотрению \textit{многошаговых методов}. В отличии от методов 
Рунге-Кутты, где мы активно использовали дополнительные стадии в вычислениях,
при применении многошаговых методов мы хотим использовать знания о решении
в предыдущие моменты времени.

\subsection{Методы Адамса}

	Методы Адамса являются представителем семейства многошаговых методов. 
	Рассмотрим основную идею построения этих методов. Проинтегрируем исходное 
	уравнение от $t_n$ до $t_{n+1}$:
	
	\begin{equation}
		\int\limits_{t_{n}}^{t_{n + 1}}\frac{du}{dt} = \int\limits_{t_{n}}
		^{t_{n + 1}}f\left(t, u\right)dt
	\end{equation}
	
	\begin{equation}
	u_{n + 1} - u_{n} = \int\limits_{t_{n}}^{t_{n + 1}}f\left(t, u\right)dt 
	\end{equation}
	
	Интеграл в правой части последнего равенства можно вычислить приближенно, 
	заменив исходную функцию $f(t,u)$ на её интерполянт по значением этой 
	функции $L_k(t)$:
	
	\begin{equation}
		u_{n + 1} - u_{n} = \int\limits_{t_{n}}^{t_{n + 1}}f\left(t, u\right)dt 
		\approx \int\limits_{t_{n}}^{t_{n + 1}}L_{k}\left(t\right) dt
	\end{equation}
	
	\subsubsection{Явный метод Адамса 2-го порядка}
	
		Строим интерполянт по значениям функции $f(t,u)$ в моменты времени $t_n
		$, $t_{n-1}$. Эти значения будем обозначать $f_n$, $f_{n-1}$.
		\begin{equation}
			L_{1} = f_{n}\cdot\frac{t - t_{n - 1}}{t_{n} - t_{n - 1}} + 
			f_{n - 1}\cdot\frac{t - t_{n}}{t_{n - 1} - t_{n}}
		\end{equation}
		
		Проинтегрируем интерполянт:
		
		\begin{equation}
			\int\limits_{t_{n}}^{t_{n + 1}}L_{1}\left(t\right)dt = 
			\frac{\Delta t}{2}\left(3\cdot f_{n} - f_{n - 1}\right)
		\end{equation}
		
		Таким образом:
		
		\begin{equation}
			\frac{y_{n + 1} - y_{n}}{\Delta t} = \frac{3}{2}f_{n} - \frac{1}{2}
			f_{n - 1}
		\end{equation}
		
	\subsubsection{Неявный метод Адамса 2-го порядка}
	
		Для построения неявного метода Адамса второго порядка рассматривается 
		интерполянт по значениям в моменты времени $t_{n+1}$, $t_n$:
		
		\begin{equation}
			L_{1} = f_{n + 1}\cdot\frac{t - t_{n}}{t_{n + 1} - t_{n}} + 
			f_{n}\cdot\frac{t - t_{n + 1}}{t_{n} - t_{n + 1}}
		\end{equation}
		
		После интегрирования интерполянта, получаем схему:
		
		\begin{equation}
			\frac{y_{n + 1} - y_{n}}{\Delta t} = \frac{1}{2}\left(
			f_{n + 1} - f_{n}\right)
		\end{equation}
		
		С этой схемой мы уже встречались, она называется \textit{неявный метод 
		трапеций}.
		
	С методами Адмаса есть несколько трудностей:
	
	\begin{enumerate}
		\item Необходимость наличия $k$ точек для начала применения метода.
		\item Возможные появления нефизических решений.
	\end{enumerate}
	
	\subsection{Многошаговый метод общего вида}
		Рассмотрим многошаговый метод общего вида:

		\begin{equation}
			y_{n} + \alpha_{-1}y_{n - 1} + \alpha_{-2}y_{n - 2} + \ldots + 
			\alpha_{-k}y_{n - k} = \beta_{1}f_{n + 1} + \beta_{0}f_{n} + 
			\beta_{-1}f_{n - 1} + \ldots + \beta_{-k}f_{n - k}
		\end{equation}
		
		С общим видом связана следующая классификация:
		
		\begin{enumerate}
			\item $\beta_{1} = 0 \rightarrow$ явный метод.
			\item $\beta_{1} \neq 0 \rightarrow$ неявный метод.
		\end{enumerate}
		
		В качестве примера рассмотрим двухшаговый неявный метод общего вида:
		
		\begin{equation}
			y_{n} + \alpha_{-1}y_{n - 1} + \alpha_{-2}y_{n - 2} = 
			\beta_{1}f_{n + 1} + \beta_{0}f_{n} + \beta_{-1}f_{n - 1}
		\end{equation}
		
		
		Для определения свободных коэффициентов $\alpha_i$, $\beta_i$ можно 
		потребовать,
		чтобы схема обладала максимальным порядком аппроксимации.
		Есть несколько вариантов исследования схемы на аппроксимацию:
		
		\begin{enumerate}
			\item Вариант основанный на разложении в ряд Тейлора. При этом для 
			упрощения выкладок стоит заменить 
			
			\begin{align}
				f_{n + 1} &\rightarrow u'_{n + 1}\\
				f_{n} &\rightarrow u'_{n} \\
				f_{n - 1} &\rightarrow u'_{n - 1}
			\end{align}
			
			\item Второй подход основан на алгебраической точности схемы. Нужно 
			потребовать, чтобы схема была точна для решений исходного ОДУ, 
			являющихся полиномами как можно более высокой степени:
			
			\begin{align}
				u &= 1 \\
				u &= t \\
				u &= t^{2}\\
				&\vdots\\
				u &= t^{n}
			\end{align}
		\end{enumerate}
		
		Рассмотрим второй вариант подробнее:

		Для полиномов 0 степени:		
		
		\begin{equation}
			u = const = 1, u' = f = 0;
		\end{equation}
		
		Подставляем в нашу схему:
		
		\begin{equation}
			1 + \alpha_{-1} + \alpha_{-2} = \beta_{1}\cdot 0 + \beta_{0}\cdot 0
			+ \beta_{-2}\cdot 0 = 0
		\end{equation}
		
		Для полиномов 1 степени:		
		
		\begin{equation}
			u = t, u' = 1;
		\end{equation}
		
		\begin{equation}
			t_{n} + \alpha_{-1}\left(t_{n} - \Delta t\right) + \alpha_{-2}
			\left(t_{n} - 2\Delta t\right) = \beta_{1} + \beta_{0} + \beta_{-1}
		\end{equation}
		
		Преобразуем:
		
		\begin{equation}
			t_{n}\left(1 + \alpha_{-1} + \alpha_{-2}\right) - \Delta t
			\alpha_{-1} - 2\Delta \alpha_{-2} = \beta_{1} + \beta_{0} + 
			\beta_{-1}
		\end{equation}
		
		Используя условие на полином 0 степени, множитель $1 + \alpha_{-1} + 
		\alpha_{-2} = 0$. Выписывая аналогичные условия для полиномов больших 
		степеней, получаем условия на коэффициенты.
		
		\section{Устойчивость}
		
		Ранее было получено, что:
		
		\begin{equation}
			\text{сходимость} \rightarrow \text{аппроксимация} + 
			\text{устойчивость}
		\end{equation}
		
		Для исследования схем будем рассматривать уравнение:
		
		\begin{equation}
			\begin{cases}
				\frac{du}{dt} = \lambda u, \Re\left(\lambda\right) \leqslant 0
				\\
				u\left(0\right) = u_{0}
			\end{cases}
		\end{equation}
		
		Данное уравнение называется \textit{Уравнение Далквиста}.
		
		Решением будет являться:
		
		\begin{equation}
			u\left(t\right) = u_{0}\exp\left(\lambda t\right)
		\end{equation}
		
		Для такого решения:
		
		\begin{equation}
			\left|u\left(t + \Delta t\right)\right|\leqslant \left|
			u\left(t\right)\right|
		\end{equation}
		
		Рассмотрим явную схему Эйлера для решения уравнение Далквиста:
		
		\begin{equation}
			\frac{y_{n + 1} - y_{n}}{\Delta t} = \lambda y_{n}
		\end{equation}
		
		\begin{equation}
			y_{n + 1} = \left(1 + \Delta t\lambda\right)y_{n}
		\end{equation}
		
		\begin{equation}
			y_{n} = \left(1 + z\right)^{n}y_{0}
		\end{equation}
		
		
		Для устойчивости мы хотим, чтобы численное решение не возрастало
		Для условия $\left|y_{n + 1}\right| \leqslant \left|y_{n}\right|
		\rightarrow \left|1 + z\right|\leqslant 1$. Данная область представляет 
		собой окружность с центром в точке (-1) в комплексной плоскости.
		
		Если $\lambda$ -- действительное, получаем:
		
		\begin{equation}
			\left|1 + \Delta t\lambda\right| \leqslant 1 \Rightarrow \Delta t 
			\leqslant \frac{2}{\left|\lambda\right|}
		\end{equation}
		%1 + \Delta t\lambda < 1
		
		Рассмотрим теперь применение неявного метода трапеций для решения 
		уравнение Далквиста:
		
		\begin{equation}
			\frac{y_{n + 1} - y_{n}}{\Delta t} = \frac{\lambda}{2}\left(
			y_{n + 1} + y_{n}\right)
		\end{equation}
		
		\begin{equation}
			y_{n + 1} = \frac{1 + \frac{\Delta t \lambda}{2}}{1 - \frac{\Delta 
			t \lambda}{2}}y_{n} = \frac{1 + \frac{z}{2}}{1 - \frac{z}{2}}y_{n}
			= R\left(z\right)y_{n}
		\end{equation}
		
		Условие на функцию $R\left(z\right)$:
		
		\begin{equation}
			\left|R\left(z\right)\right|\leqslant 1
		\end{equation}
		
		Решение данного неравенства -- вся левая комплексная полуплоскость, то 
		есть метод устойчив при любом выборе $\Delta t$.
		
		Еще один пример -- неявный метод Эйлера:
		
		\begin{equation}
			\frac{y_{n + 1} - y_{n}}{\Delta t} = \lambda y_{n + 1}
		\end{equation}
		
		\begin{equation}
			y_{n + 1} = \frac{1}{1 - z}y_{n}
		\end{equation}
		
		Условие на z:
		
		\begin{equation}
			\left|\frac{1}{1 - z}\right| \leqslant 1
		\end{equation}
		
		Область устойчивости -- внешность шара радиуса $1$ с центром в точке 
		$1$.
		
		Для методов Рунге-Кутты есть явное выражение для функции устойчивости 
		$R(z)$.
		
		Обозначим таблицу Бутчера следующим образом:
		
		\begin{table}[h!]
\centering
\begin{tabular}{l|l}
 $\vec{c}$ & $A$       \\ \hline
\vspace{1mm} &\vspace{1mm} $\vec{b}^T$
\end{tabular}
\end{table}

 		В таком случае, получаем следующее выражение:
 		
 		\begin{equation}
 			R\left(Z\right) = \frac{\det\left(\mathbb{I} - zA + z\vec{e}\cdot
 			\vec{b}^{T}\right)}{\det\left(\mathbb{I} - zA\right)}
 		\end{equation}
 		
 		Для явных методов Рунге-Кутты выражение для функции $R(z)$ значительно 
 		упрощается, если Порядок аппроксимации = количество стадий. В этом 
 		случае:
 		
 		\begin{equation}
 			R\left(Z\right) = \sum\limits_{k = 0}^{N_{k}}\frac{z^{k}}{k!},
 		\end{equation} 
 		где $N_k$ - количество стадий (порядок аппроксимации) метода. То есть 
 		функции $R(z)$ в данном случае -- обрезанное разложение экспоненты в ряд 
 		Тейлора.
 		
\end{document}