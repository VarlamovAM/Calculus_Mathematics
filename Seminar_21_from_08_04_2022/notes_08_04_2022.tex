\documentclass[10pt,a4paper]{article}
\usepackage[utf8]{inputenc}
\usepackage[russian]{babel}
\usepackage[OT1]{fontenc}
\usepackage{amsmath}
\usepackage{amsfonts}
\usepackage{amssymb}
\usepackage[dvipsnames]{xcolor}
\usepackage{graphicx}
\graphicspath{{Images/}}
\usepackage[left=2cm,right=2cm,top=2cm,bottom=2cm]{geometry}
\usepackage{calc}
\usepackage{wrapfig}
\usepackage{setspace}
\usepackage{indentfirst}
\usepackage{subfigure}
\usepackage{multirow}
\usepackage{amsfonts}
\usepackage{hyperref}
\hypersetup{
    pdfstartview=FitH,  
    linkcolor=black,
    urlcolor=red, 
    colorlinks=true,
    citecolor=blue}
\usepackage{tikz}
\usetikzlibrary{ decorations.markings}

\title{Семинар 21}
\date{\today}

\author{Варламов Антоний Михайлович}

\begin{document}
	\maketitle
	
	\section{Задача XIII.7.3}
	
	Рассмотрим схему 
	
	\begin{equation}
		\frac{y^{n + 1}_{m} - y^{n}_{m}}{\Delta t} = \frac{y^{n}_{m - 1} - 2
		y^{n + 1}_{m} + y^{n}_{m + 1}}{h^{2}}
	\end{equation}
	
	Вспомним условие устойчивости для явной схемы: (на примере Явной схемы 
	Эйлера)
	
	\begin{equation}
		\sigma^{2} = \frac{\Delta t}{h} \leqslant \frac{1}{2}
	\end{equation}
	
	Для неявной схемы:
	
	\begin{equation}
		\sigma^{2} \leqslant 0
	\end{equation}
	
	\subsection{Аналогичность схеме Эйлера}
	
	
	Преобразуем:
	\begin{equation}
		\frac{y^{n + 1}_{m} - y^{n}_{m}}{\Delta t} = \frac{y^{n}_{m - 1} - 2
		y^{n + 1}_{m} + y^{n}_{m + 1}}{h^{2}} - \frac{2y^{n}_{m}}{h^{2}} + 
		\frac{2y^{n}_{m}}{h^{2}}
	\end{equation}
	
	Получим:
	
	\begin{equation}
		\frac{y^{n + 1}_{m} - y^{n}_{m}}{\Delta t} = \frac{y^{n}_{m - 1} - 2
		y^{n}_{m} + y^{n}_{m + 1}}{h^{2}} - \frac{2}{h^{2}}\left(
		y^{n + 1}_{m} - y_{m}^{n}\right)
	\end{equation}
	
	\begin{equation}
		\left(y_{m}^{n + 1} - y_{m}^{n}\right)\left(\frac{1}{\Delta t} + 
		\frac{2}{h^{2}}\right) = \frac{y^{n}_{m - 1} - 2
		y^{n}_{m} + y^{n}_{m + 1}}{h^{2}}
	\end{equation}
	
	\begin{equation}
		\frac{1}{\Delta t'} = \left(\frac{1}{\Delta t} + \frac{2}{h^{2}}\right)
	\end{equation}
	
	\begin{equation}
		\Delta t' = \frac{\Delta t\cdot h^{2}}{2\Delta t + h^{2}} = 
		\Delta t\left(\frac{1}{1 + \frac{2\Delta t}{h^{2}}}\right) \leqslant 
		\Delta t
	\end{equation}
	
	\subsection{Исследование на устойчивость}
	
	Условие устойчивости для данной схемы:
	
	\begin{equation}
		\frac{\Delta t'}{h^{2}}\leqslant\frac{1}{2}
	\end{equation}
	
	\begin{equation}
		\frac{\Delta t}{h^{2}}\left(\frac{1}{1 + \frac{2\Delta t}{h^{2}}}\right)
		\leqslant \frac{1}{2}
	\end{equation}
	
	\begin{equation}
		\sigma^{2}\cdot\left(\frac{1}{1 + 2\sigma^{2}}\right) \leqslant \frac{1}
		{2}
	\end{equation}
	
	Данное условие выполняется для любых $\sigma$. Значит, схема Алена-Чена 
	безусловно устойчива
	
	\subsection{Исследование на аппроксимацию}
	
	Вектор невязки:
	
	\begin{equation}
		r^{n}_{m} = \frac{\Delta t}{2}y'_{t} + y^{\left(IV\right)}_{x}\cdot
		\frac{\Delta x^{2}}{12} + \frac{2\Delta t}{\Delta x^{2}}\cdot y'_{t} 
		\ldots
	\end{equation}
	
	Значит, необходимо, чтобы:
	
	\begin{equation}
		\frac{\Delta t}{\Delta x^{2}}\rightarrow 0
	\end{equation}
	
	Как правило, для подобных схем используется обозначение:
	
	\begin{equation}
		O\left(\Delta t, \Delta x^{2}, \frac{\Delta t}{\Delta x^{2}}\right)
	\end{equation}
	
	Подобными схемами являются Схема Ричардсона или исследуемая Схема Алана-Чена
	\vspace{0.5cm}\\
	\textbf{\textit{Небольшое Лирическое Отступление}}
	\vspace{0.5cm}\\
	В реальности существует список методов, использующихся для построения 
	аппроксимаций:
	
	\begin{enumerate}
		\item Finite element Method
		\item Finite volume Method
		\item Spectral element Method
		\item Discontious Galerkin Method
	\end{enumerate}

	\section{Задача XIII.9.3}
	
	\begin{equation}
		\frac{y^{n + 1}_{m} - y^{n}_{m}}{\tau} = \frac{y^{n}_{m - 1} - 2
		y^{n}_{m} + y^{n}_{m + 1}}{h^{2}}
	\end{equation}
	
	\begin{equation}
		\psi'_{t} = \psi''_{xx}
	\end{equation}
	
	\begin{equation}
		\psi''_{tt} = \psi'''_{xxt} = \psi^{\left(IV\right)}_{x}
	\end{equation}
	
	Значит, получаем:
	
	\begin{equation}
	\left(\frac{\Delta t}{2} - \frac{\Delta x^{2}}{12}\right)\psi^{\left(IV
	\right)}_{x}
	\end{equation}
	
	\begin{equation}
		\Delta t = \frac{\Delta x^{2}}{6} \rightarrow O\left(\Delta t^{2},
		\Delta x^{4}\right)
	\end{equation}
	
	\section{Задача XIII.9.4}
	
	Аппроксимируемое уравнение:
	
	\begin{equation}
		\psi'_{t} = \psi''_{x}
	\end{equation}
	
	Выпишем общий вид Схемы:
	
	\begin{equation}
		a\cdot\psi_{m - 1}^{n + 1} + b\cdot\psi_{m}^{n + 1} + c\cdot\psi_{m + 1}
		^{n + 1} + d\cdot\psi^{n}_{m - 1} + e\cdot\psi^{n}_{m} + f\cdot
		\psi^{n}_{m + 1}
	\end{equation}
	
	из разложения в ряд Тейлора, получаем:
	
	\begin{equation}
		\begin{cases}
			a + b + c + d + e + f = 0
			\\
			\left(a + b + c\right)dt = 1
			\\
			\frac{\left(a + c + d + f\right)dx^{2}}{2} = 1
			\\
			a + c + d - f = 0
			\\
			\left(a + c + d - f\right)dx^{2} + 6dt\left(a - c\right) = 0
			\\
			5\left(a + c + d + f\right)dx^{4} + 60\left(a + c\right)dx^{2}dt + 
			60\left(a + b + c\right)dt^{2} = 0
		\end{cases}
	\end{equation}
	
	Для удобства введем обозначение:
	
	\begin{equation}
		\left(\mathbb{L}\psi\right)^{n}_{m} = \frac{\psi^{n}_{m - 1} - 2\psi^{n}
		_{m} + \psi^{n}_{m + 1}}{\Delta x^{2}}
	\end{equation}
	
	\begin{equation}
		\left(\delta_{t}\psi\right)_{m}^{n} = \frac{\psi^{n + 1}_{m} - \psi_{m}
		^{n}}{\Delta t}
	\end{equation}
	
	В таком случае, получим схему в виде:
	
	\begin{equation}
		\frac{1}{12}\left(\delta_{t}\psi\right)_{m + 1}^{n} + 
		\frac{5}{6}\left(\delta_{t}\psi\right)_{m}^{n} + 
		\frac{1}{12}\left(\delta_{t}\psi\right)_{m - 1}^{n} = 
		\frac{1}{2}\left(\left(\mathbb{L}\psi\right)^{n + 1}_{m} + 
		\left(\mathbb{L}\psi\right)^{n}_{m}\right)
	\end{equation}

	\section{Задача №3}
	
	\begin{equation}
		\frac{\psi_{m}^{n + 1} - \psi_{m}^{n}}{\Delta t} = \left(1 - \xi\right)
		\left(\mathbb{L}\psi\right)^{n + 1}_{m} + \xi\left(\mathbb{L}\psi
		\right)^{n}_{m}
	\end{equation}
	
	\section{Задача №4}
	
	\begin{equation}
		\frac{\partial \psi }{\partial t} + c\frac{\partial \psi}{\partial x}=0
	\end{equation}
	
	Рассмотрим различные схемы:
	
	\begin{enumerate}
		\item $
			\frac{\psi^{n + 1}_{m} + \psi_{m}^{n}}{\Delta t} + 
			c\frac{\psi_{m + 1}^{n} - \psi_{m - 1}^{n}}{2\Delta x} = 0
		$
		\item $
			\frac{\psi^{n + 1}_{m} + \psi_{m}^{n}}{\Delta t} + 
			c\frac{\psi_{m}^{n} - \psi_{m - 1}^{n}}{\Delta x} = 0
		$
		\item $
			\frac{\psi^{n + 1}_{m} + \psi_{m}^{n}}{\Delta t} + 
			c\frac{\psi_{m + 1}^{n} - \psi_{m}^{n}}{\Delta x} = 0
		$
		
	\end{enumerate}


	Рассмотрим данные схемы:
	
	\begin{enumerate}
		\item \begin{equation}
			\psi^{n}_{m} = \lambda^{n}\cdot e^{i\alpha m}
		\end{equation}
		
		\begin{equation}
			\frac{\lambda - 1}{\Delta t} + c\frac{e^{i\alpha} - e^{-i\alpha}}
			{2\Delta x} = 0
		\end{equation}
		
		\begin{equation}
			\lambda = 1 + \frac{c\Delta t}{\Delta x}i\sin\alpha
		\end{equation}
		
		\begin{equation}
			\left|\lambda\right| = 1 + \frac{c^{2}\Delta t^{2}}{\Delta x^{2}}
			\sin^{2}\alpha \geqslant 1
		\end{equation}
		
		Значит данная схема является \textit{неустойчивой}
	\end{enumerate}
\end{document}