\documentclass[10pt,a4paper]{article}
\usepackage[utf8]{inputenc}
\usepackage[russian]{babel}
\usepackage[OT1]{fontenc}
\usepackage{amsmath}
\usepackage{amsfonts}
\usepackage{amssymb}
\usepackage[dvipsnames]{xcolor}
\usepackage{graphicx}
\graphicspath{{Images/}}
\usepackage[left=2cm,right=2cm,top=2cm,bottom=2cm]{geometry}
\usepackage{calc}
\usepackage{wrapfig}
\usepackage{setspace}
\usepackage{indentfirst}
\usepackage{subfigure}
\usepackage{multirow}
\usepackage{amsfonts}
\usepackage{hyperref}
\hypersetup{
    pdfstartview=FitH,  
    linkcolor=black,
    urlcolor=red, 
    colorlinks=true,
    citecolor=blue}
\usepackage{tikz}
\usetikzlibrary{ decorations.markings}

\title{Семинар 7}
\date{\today}



\author{Варламов Антоний Михайлович}

\begin{document}
	\maketitle
	
	\section{Метод сжимающих отображений}
	
	\begin{equation}
		f\left(x\right) = 0
	\end{equation}
	
	\begin{equation}
		x^{6} - 5x - 2 = 0
	\end{equation}
	
	\begin{enumerate}
		\item Локализуем корень:
			\begin{equation}
				f\left(1\right) = -6, \ \ f\left(2\right) = 52
			\end{equation}
			
			\begin{equation}
				\Rightarrow x^{*} \in \left[1, 2\right]
			\end{equation}
		\item $f\left(x\right) \rightarrow x = g\left(x\right)$
		
		Есть несколько возможных вариантов такого перехода:
		
		\begin{enumerate}
			\item $x = \frac{x^{6} - 2}{5}$
			\item $x = \sqrt[6]{5x + 2}$
			\item $x = \frac{5}{x^{4}} + \frac{2}{x^{5}}$
		\end{enumerate}
		
		Для $x = \frac{x^{6} - 2}{5}$:
		
		\begin{equation}
			x^{k + 1} - \text{сжимающее?}
		\end{equation}
		
		\begin{equation}
			\left|g'\left(x\right)\right| = \left|\frac{1}{5} \cdot 6x^{5}
			\right|
		\end{equation}
		
		\begin{equation}
			\Rightarrow g'\left(1\right) = \frac{6}{5} > 1
		\end{equation}
		
		Для $x = \sqrt[6]{5x + 2}$:
		
		\begin{equation}
			g\left(x\right) = \sqrt[6]{5x + 2}
		\end{equation}
		
		\begin{equation}
			g'\left(x\right) = \frac{5}{6} \frac{1}
			{\left(5x + 2\right)^{\frac{5}{6}}}
		\end{equation}
		
		\begin{equation}
			\left|g'\left(x\right)\right| < 1
		\end{equation}
		
		\begin{equation}
			g\left(2\right) < 2, \ \ 2 > g\left(1\right) > 1
		\end{equation}
		
		Для $x = \frac{5}{x^{4}} + \frac{2}{x^{5}}$:
		
		\begin{equation}
			g'\left(x\right) = \frac{-20}{x^{5}} - \frac{10}{x^{6}}
		\end{equation}
		
		\begin{equation}
			\left|g'\left(1\right)\right| = 30
		\end{equation}
		
		\item Определим количество итераций для заданной точности:
		
		\begin{equation}
			\left|x^{*} - x^{k}\right| \leqslant \varepsilon	
		\end{equation}				
		
		\begin{equation}
			q^{k} < \frac{\varepsilon}{\left|x^{1} - x^{0}\right|}
		\end{equation}
	\end{enumerate}
	
	\section{Метод Ньютона}
	
		Основан на линейном приближении функции.
		
		\begin{equation}
			f_{L}\left(x\right) = f\left(x_{k}\right) + f'\left(x_{k}\right)
			\left(x - x_{k}\right) = 0
		\end{equation}
		
		\begin{equation}
			x_{k + 1} = x_{k} + \frac{f\left(x_{k}\right)}{f'\left(x_{k}\right)}
		\end{equation}
		
		\begin{equation}
			\left|x^{*} - x^{k + 1}\right| \leqslant C 
			\left|x^{*} - x^{k}\right|^{2}
		\end{equation}
		
		В данном методе могут возникнуть проблемы связанные с наличием 
		производной в формуле. Приблизив производную численным 
		дифференцированием можно получить метод секущих.
		
		Метод Ньютона является \textit{сжимающим} отображением.
		
	\section{Многомерный случай}
	
		Постановка задачи:
		
		\begin{equation}
			\vec{F}\left(\vec{x}\right) = \vec{0}
		\end{equation}
		\subsection{Метод Сжимающих отображений для многомерного случая}
		Пусть $Y$ -- полное метрическое пространство, $\rho\left(x, y\right),
		\rho\left(x, y\right) = \parallel x - y\parallel$, $\Omega \subset Y$ --
		замкнутое, $G$ -- сжимающее при $\vec{x} \in \Omega$
		
		\begin{enumerate}
			\item $G: \Omega \rightarrow \Omega$
			\item $\rho\left(G\left(\vec{x}\right),
			G\left(\vec{y}\right)\right)\leqslant q\rho\left(\vec{x}, \vec{y}
			\right), q < 1$ 
		\end{enumerate}
		
		Достаточное условие сжимающего отображения:
		
		\begin{enumerate}
			\item $G: \Omega \rightarrow \Omega$
			\item $\parallel G'\parallel \leqslant q < 1$ G' -- матрица Якоби
		\end{enumerate}
		
		Лучше всего брать вторую норму матрицы Якоби или рассматривать ее 
		спектральный радиус.
		
		Рассмотрим пример:
		
		\begin{equation}
			\begin{cases}
				x + 3\ln x - y^{2} = 0
				\\
				2x^{2} - xy - 5x + 1 = 0
			\end{cases}
		\end{equation}
		
		\begin{enumerate}
			\item Локализуем корень:
			
			\begin{equation}
				\Omega: x^{*} \in \left[3, 4\right], y^{*} \in \left[2, 3\right]
			\end{equation}
			
			\begin{equation}
				\Omega = \left[3, 4\right] \times \left[2, 3\right]
			\end{equation}
			
			\item 
			\begin{equation}
				\begin{cases}
					x = y^{2} - 3\ln x
					\\
					y = 2x - 5 + \frac{1}{x}
				\end{cases}
			\end{equation}
			
			\begin{equation}
				\begin{cases}
					x_{k + 1} = y_{k}^{2} - 3\ln x_{k}
					\\
					y_{k + 1} = 2x_{k} - 5 + \frac{1}{x_{k}}
				\end{cases}
			\end{equation}
			
			\item Определим матрицу Якоби:
			
			\begin{equation}
				\parallel G'\left(x, y\right)\parallel_{1} = \max 
				\left(\left|2y\right|, \left|\frac{3}{x}\right| + 
				\left|2 - \frac{1}{x^{2}}\right|\right) = \left|2y\right| > 1
			\end{equation}
			
			Придется рассматривать другой вариант:
			
			\begin{equation}
				\begin{cases}
					y = \sqrt{x + 3\ln x}
					\\
					x = \sqrt{\frac{xy + 5x - 1}{2}}
				\end{cases}
			\end{equation}
			
			В таком случае $\parallel G'\left(x, y\right)\parallel < 1$
		\end{enumerate}
		
		\subsection{Метод Ньютона для многомерного случая}
		
		\begin{equation}
			\vec{x}_{k + 1} = \vec{x}_{k} - 
			\left[F'\left(\vec{x}\right)\right]^{-1}\cdot F\left(\vec{x}\right)
		\end{equation}
		
	\section{Минимизация функций}
	
		Постановка задачи: Найти $\min\limits_{x \in \left[a, b\right]} f\left(x
		\right) \leftrightarrow \frac{\partial f}{\partial x} = 0$
		
		\subsection{Метод дихотомии}
		
			Данный метод работает для \textit{унимодальных} функций.
			
			Функция является \textit{унимодальной}, если $\exists \ \xi \in 
			\left[a, b\right]: f\left(x_{1}\right) > f\left(x_{2}\right) 
			\ \forall x_{1}, x_{2}: x_{1} < x_{2} < \xi, 
			f\left(x_{1}\right) > f\left(x_{2}\right) \forall x_{1}, x_{2}:
			x_{1} > x_{2} > \xi$
			
			Пусть $x^{*}$ -- минимум функции. Локализуем минимум:
			
			\begin{equation}
				x^{*} \in \left[a, b\right], a_{0} = a, b_{0} = b, c_{0} = 
				\frac{a_{0} + b_{0}}{2}
			\end{equation}
			
			определим точку $y_{k} = \frac{a_{k} + c_{k}}{2}$
			
			Возможные варианты:
			
			\begin{enumerate}
				\item $f\left(y_{k}\right) < f\left(c_{k}\right)$:
				
				\begin{equation}
					\begin{cases}
						a_{k + 1} = a_{k}
						\\
						b_{k + 1} = c_{k}
						\\
						c_{k + 1} = y_{k}
					\end{cases}
				\end{equation}
				
				\item $f\left(y_{k}\right) > f\left(c_{k}\right)$:
				
				\begin{equation}
					z_{k} = \frac{b_{k} + c_{k}}{2}
				\end{equation}
				\begin{enumerate}
					\item $f\left(z_{k}\right) > f\left(c_{k}\right)$:
					
					\begin{equation}
						\begin{cases}
							a_{k + 1} = y_{k}
							\\
							b_{k + 1} = z_{k}
							\\
							c_{k + 1} = c_{k}
						\end{cases}
					\end{equation}
					\item $f\left(z_{k}\right) < f\left(c_{k}\right)$:
					
					\begin{equation}
						\begin{cases}
							a_{k + 1} = c_{k}
							\\
							b_{k + 1} = b_{k}
							\\
							c_{k + 1} = z_{k}
						\end{cases}
					\end{equation}
				\end{enumerate}
			\end{enumerate}
		\subsection{Метод градиентного спуска}
		
		\begin{equation}
			\vec{x}_{k + 1} = \vec{x}_{k} - \alpha \nabla 
			\Phi\left(\vec{x}_{k}\right)
		\end{equation}
		
		\subsection{Метод наискорейшего спуска}
		
		\begin{equation}
			\vec{x}_{k + 1} = \vec{x}_{k} - \alpha \nabla 
			\Phi\left(\vec{x}_{k}\right)
		\end{equation}
		
		\begin{equation}
			\alpha_{k} = argmin\left(\Phi\left(\vec{x}_{k} - 
			\alpha_{k}\nabla \Phi\left(\vec{x}_{k}\right)\right)\right)
		\end{equation}
		
		\subsection{Метод покоординатного спуска}
		
		\begin{equation}
			\begin{cases}
				x_{1}^{k+1} = argmin_{x}\Phi\left(x, x_{2}^{k}, \ldots, 
			x_{n}^{k}\right)
			\\
				x_{2}^{k+1} = argmin_{x}\Phi\left(x_{1}^{k}, x, \ldots, 
			x_{n}^{k}\right)
			\\
			\vdots
			\\
				x_{n}^{k+1} = argmin_{x}\Phi\left(x_{1}^{k}, x_{2}^{k}, \ldots, 
			x\right)
			\end{cases}
		\end{equation}
\end{document}