\documentclass[10pt,a4paper]{article}
\usepackage[utf8]{inputenc}
\usepackage[russian]{babel}
\usepackage[OT1]{fontenc}
\usepackage{amsmath}
\usepackage{amsfonts}
\usepackage{amssymb}
\usepackage[dvipsnames]{xcolor}
\usepackage{graphicx}
\graphicspath{{Images/}}
\usepackage[left=2cm,right=2cm,top=2cm,bottom=2cm]{geometry}
\usepackage{calc}
\usepackage{wrapfig}
\usepackage{setspace}
\usepackage{indentfirst}
\usepackage{subfigure}
\usepackage{multirow}
\usepackage{amsfonts}
\usepackage{hyperref}
\hypersetup{
    pdfstartview=FitH,  
    linkcolor=black,
    urlcolor=red, 
    colorlinks=true,
    citecolor=blue}
\usepackage{tikz}
\usetikzlibrary{ decorations.markings}

\title{Семинар 20}
\date{\today}

\author{Варламов Антоний Михайлович}

\begin{document}
	\maketitle
	
	На прошлом занятии было сделано утверждение:
	$$
	\text{\textit{Аппроксимация} + \textit{Устойчивость }} \rightarrow 
	\text{\textit{Сходимость}}
	$$
	Кроме того, был затронут \textit{метод линий}.
	
	При обсуждении вопросов устойчивости активно используется 
	\textit{Спектральный признак Неймана}:
	
	\begin{equation}
		\psi_{m}^{n} = \lambda^{n}\cdot e^{i\alpha m}
	\end{equation}
	
	\begin{equation}
		\Delta t, \Delta x: \left|\lambda\right| \leqslant 1 \ \forall \ \ 
		\alpha
	\end{equation}
	
	Рассмотрим аппроксимацию по пространству:
	
	\begin{equation}
		\frac{\partial^{2}\psi}{\partial x^{2}} \approx \frac{\psi_{m + 1}
		\left(t\right) - 2\psi_{m} + \psi_{m - 1}\left(t\right)	\left(t\right)}
		{\Delta x^{2}}
	\end{equation}
	
	\begin{equation}
		\frac{d \psi_{m}\left(t\right)}{dt} = a\left(t\right)\cdot e^{i\alpha m}
	\end{equation}
	
	\begin{equation}
		\frac{d a\left(t\right)}{dt}\cdot e^{i\alpha m} = \frac{e^{i\alpha m}
		}{\Delta x^{2}}\cdot\left(2\cos\alpha - 2\right) = \frac{-4}{\Delta 
		x^{2}} \cdot sin^{2}\left(\frac{\alpha}{2}\right)a\left(t\right)
	\end{equation}
	
	\begin{equation}
		\Rightarrow \frac{da\left(t\right)}{dt} = \frac{-4}{\Delta x^{2}}\cdot
		\sin^{2}\left(\frac{\alpha}{2}\right)a\left(t\right) = 
		\lambda\cdot a\left(t\right)
	\end{equation}
	
	Получаем уравнение Далквиста:
	
	\begin{equation}
		\frac{da\left(t\right)}{dt} = \lambda a\left(t\right)
	\end{equation}
	
	\begin{equation}
		\lambda \leqslant 0, \lambda \in \mathbb{R},\ \ \ \left| 1 + \Delta t
		\lambda	\right| \leqslant 1
	\end{equation}
	
	\begin{equation}
		\Rightarrow \Delta t \leqslant \frac{2}{\max\limits_{\alpha}\lambda
		\left(\alpha\right)}
	\end{equation}
	
	Значит для данного примера:
	
	\begin{equation}
		\Delta t\leqslant \frac{2\cdot\Delta x^{2}}{4} = \frac{\Delta x^{2}}{2}
		\ \Rightarrow \ \ \frac{\Delta t}{\Delta x^{2}} \leqslant \frac{1}{2}
	\end{equation}
	
	Рассмотрим так называемый метод "Чехарды" (leapfrog):
	
	\begin{equation}
		\frac{\psi^{n + 1}_{m} - \psi^{n - 1}_{m}}{2\Delta t} = 
		\frac{\psi_{m + 1}^{n} - 2\psi_{m}^{n} + \psi_{m - 1}^{n}}{\Delta x^{2}}
	\end{equation}
	
	Опуская вычисления:
	
	\begin{equation}
		\left|Im \lambda\right|\cdot \Delta t \leqslant 1
	\end{equation}
	
	Значит, данный метод неприменим к данной задаче.
	
	\subsection{Метод Неопределенных коэффициентов}
	
	Рассмотрим уравнение:
	
	\begin{equation}
		\frac{\partial \psi}{\partial t} + \gamma\frac{\partial \psi}{\partial 
		x} = 0
	\end{equation}
	
	Рассмотрим пространственно временную сетку: $x_{m}, t_{n}$
	
	Построим разностуню схему в виде:
	
	\begin{equation}
		a\psi^{n + 1}_{m} + b\psi^{n}_{m - 1} + c\psi^{n}_{m} + d\psi_{m + 1}
		^{n} = 0
	\end{equation}
	
	Потребуем заданную аппроксимацию:
	
	\begin{equation}
		\begin{cases}
			a + b + c + d = 0;
			\\
			a\cdot dt = 1;
			\\
			-\left(b - d\right)\cdot dx = \gamma;
			\frac{\left(b + d\right)dx^{2}}{2} = 0;
		\end{cases}
		\Rightarrow 
		\begin{cases}
			a = \frac{1}{dt}
			\\
			c = \frac{-1}{dt}
			\\
			b = -d
			\\
			b = -\frac{\gamma}{2dx}
		\end{cases}
	\end{equation}
	
	Значит, получаем схему:
	
	\begin{equation}
		\frac{\psi_{m}^{n + 1} - \psi_{m}^{n}}{\Delta t} + \gamma
		\frac{\psi^{n}_{m + 1} - \psi^{n}_{m - 1}}{2\Delta x} = 0 \rightarrow 
		O\left(dt, dx^{2}\right)
	\end{equation}
	
	Но, для данного уравнения:
	
	\begin{equation}
		\psi'_{t} = -\gamma\psi'_{x} \Rightarrow \psi''_{tt} = -\gamma
		\left(\psi'_{x}\right)'_{t} = \gamma^{2}\cdot\psi''_{xx}
	\end{equation}
	
	Получаем:
	
	\begin{equation}
		\frac{a\psi''_{tt}}{2}\cdot dt^{2} + \frac{\psi''_{xx}\left(b + d
		\right)dx^{2}}{2} = \psi''_{xx}\left(\frac{\gamma^{2} a}{2}dt^{2} + 
		\frac{dx^{2}}{2}\right)
	\end{equation}
	
	В таком случае, схема приобретет вид:
	
	\begin{equation}
		\frac{\psi^{n + 1}_{m} - \psi^{n}_{m}}{\Delta t} + 
		\gamma\frac{\psi^{n}_{m + 1} - \psi^{n}_{m - 1}}{2\Delta x} - 
		\frac{\gamma^{2}\Delta t}{2}\left(\psi^{n}_{m + 1} - 2\psi_{m}^{n} + 
		\psi_{m - 1}^{n}\right) = 0
	\end{equation}
	
	Данная схема имеет название \textit{схема Лакса Вендроффа}
\end{document}