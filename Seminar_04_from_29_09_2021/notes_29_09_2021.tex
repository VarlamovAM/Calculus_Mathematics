\documentclass[10pt,a4paper]{article}
\usepackage[utf8]{inputenc}
\usepackage[russian]{babel}
\usepackage[OT1]{fontenc}
\usepackage{amsmath}
\usepackage{amsfonts}
\usepackage{amssymb}
\usepackage[dvipsnames]{xcolor}
\usepackage{graphicx}
\graphicspath{{Images/}}
\usepackage[left=2cm,right=2cm,top=2cm,bottom=2cm]{geometry}
\usepackage{calc}
\usepackage{wrapfig}
\usepackage{setspace}
\usepackage{indentfirst}
\usepackage{subfigure}
\usepackage{multirow}
\usepackage{amsfonts}
\usepackage{hyperref}
\hypersetup{
    pdfstartview=FitH,  
    linkcolor=black,
    urlcolor=red, 
    colorlinks=true,
    citecolor=blue}
\usepackage{tikz}
\usetikzlibrary{ decorations.markings}

\title{Семинар 4}
\date{\today}

\author{Варламов Антоний Михайлович}

\begin{document}
	
	\maketitle	
	
	\section{СЛАУ}
	
		Итерационные методы:
		
		\begin{equation}
			x_{k + 1} = x_{k} + P^{-1}\cdot r_{k}
		\end{equation}
		
		где $r_{k} = b - Ax_{k}$ -- вектор невязки.
		
		$e_{k} = x - x_{k}$, В таком случае: $Ae_{k} = r_{k} \rightarrow r_{k}
		= 0 \rightarrow e_{k} = 0$
		
		Необходимое и достаточное условие сходимости:
		
		\begin{equation}
			\rho\left(\mathbb{I} - P^{-1}A\right) < 1
		\end{equation}
		
		Достаточное условие:
		
		\begin{equation}
			\parallel \mathbb{I} - P^{-1}A \parallel < 1
		\end{equation}
		
		Кроме того, выполняется соотношение:
		
		\begin{equation}
			\parallel e_{k + 1} \parallel = \parallel 
			\left(\mathbb{I} - P^{-1}A\right)e_{k}\parallel \leqslant
			\parallel S \parallel \cdot \parallel e_{k} \parallel 
		\end{equation}
		
		\begin{equation}
			\parallel e_{k + 1} \parallel \leqslant q\cdot 
			\parallel e_{k} \parallel \leqslant q^{2} \parallel e_{k - 1}
			\parallel \leqslant \ldots
		\end{equation}
		
		\begin{equation}
			\rho\left(A\right) \leqslant \parallel A \parallel
		\end{equation}
		
		\begin{equation}
			\parallel e_{k + 1} \parallel_{2} \leqslant 
			\left|\lambda_{\max} \left(S \right)\right| \leqslant 
			\parallel e_{k} \parallel_{2}
		\end{equation}
		
		Метод Ричардсона:
		
		$P^{-1} = \tau$
		
		Работает для $А = A^{T}$ > 0; $0 < \tau < \frac{2}{\lambda_{\max}}$
		
		$$\tau_{\text{оптим}} = \frac{2}{\lambda_{\max} + \lambda_{\min}}$$
		
		$$q_{\text{оптим}} = \frac{\mu - 1}{\mu + 1} = \frac{\lambda_{\max} -
		 \lambda_{\min}}{\lambda_{\max} + \lambda_{\min}}$$
		 
		Метод Якоби:
		 
		$P^{-1} = D^{-1}, \ \ \ A = D + L + U$
		
		$$x_{k + 1} = x_{k} + D^{-1}r_{k}$$
		
		$$\left|a_{ii}\right| > \sum\limits_{j\neq i}, \left|a_{ij}\right|$$
		
		Есть также метод Гаусса-Зейделя
		
		\newpage	
			
		\section{Спектральные задачи}
		
		Общая постановка: $Ax = \lambda x$
		
		\begin{equation*}
			Ax = \lambda x \rightarrow \sum\limits_{i = 0}^{n - 1}b_{i}
			\lambda^{i} = 0
		\end{equation*}
		
		Как правило интерес представляет не полный спектр, а частные проявления:
		
		\begin{enumerate}
			\item $\max \left|\lambda_{i}\right|$
			\item $\max\lambda_{i}$
			\item $\min \left|\lambda_{i}\right|$
			\item $\min \lambda_{i}$
		\end{enumerate}
		
		Пусть $\lambda_{i} \in \mathbb{R}$, а также:
		
		\begin{equation}
			\lambda_{1} > \lambda_{2} \geqslant \lambda_{3} \geqslant \ldots
		\end{equation}
		
		\begin{equation}
			A^{k} x = \sum \lambda_{i}^{k}\alpha_{i}z_{i} = 
			\lambda_{1}^{k}\alpha_{1}\left(z_{1} + 
			\left(\frac{\lambda_{2}}{\lambda_{1}}\right)^{2}
			\frac{\alpha_{2}}{\alpha_{1}} z_{2} + \ldots\right)
			\rightarrow \lambda_{1}^{k}\alpha_{1}z_{1}
		\end{equation}
		
		Выше описан так называемый степенной метод.
		
		Идейно его можно описать как:
		
		\begin{eqnarray}
			x_{0} = a \\
			for k = 1, \ldots \\
			x_{k} = A\cdot x_{k-1}\\
			\lambda_{k} = \frac{\left(x_{k}, x_{k - 1}\right)}
			{\left(x_{k - 1}, x_{k - 1}\right)}
		\end{eqnarray}
		
		Однако такой метод очень быстро переполнит норму вектора. Для избежания 
		такого исхода необходимо проводить нормировку вектора на каждой 
		итерации.
		
		
		Рассмотрим практическую задачу:
		
		\begin{equation}
			\begin{cases}
				T''_{xx} = f\left(x\right)
				\\
				T\left(0\right) = \alpha
				\\
				T\left(2\pi\right) = \beta
			\end{cases}
		\end{equation}
		
		Идея решения -- замена производных численными производными.
		
		\begin{equation}
			\frac{T_{i + 1} - 2T_{i} + T_{i - 1}}{h^{2}} = f_{i}, i \in 
			\left[2, N - 2\right]
		\end{equation}
		
		Для узла 1: $\frac{T_{2} - 2T_{1}}{h^{2}} = f_{1} - \frac{\alpha}{h}$, 
			
		Для  узла N-1: $\frac{T_{n} - 2T_{N-1}}{h^{2}} = 
		f_{n-2} - \frac{\beta}{h}$
		
		Введем вектора:
		
		\begin{equation}
			f = \left(f_{1} - \frac{\alpha}{h^2}, f_{2}, \ldots, f_{N - 2},
			f_{N - 1} - \frac{\beta}{h^{2}}\right)^{T}
		\end{equation}
		
		В таком случае матрица будет иметь вид:
		
		\begin{equation}
			\begin{pmatrix}
				-2 & 1 & 0 & 0 & \ldots & 0 \\
				1 & -2 & 1 & 0 & \ldots & 0 \\
				0 & 1 & -2 & 1 & \ldots & 0 \\
				\vdots & \vdots & \vdots & \vdots & \ddots & \vdots\\
				0 & 0 & 0 & \ldots & 1 & 2\\
			\end{pmatrix}
		\end{equation}
				
\end{document}