\documentclass[10pt,a4paper]{article}
\usepackage[utf8]{inputenc}
\usepackage[russian]{babel}
\usepackage[OT1]{fontenc}
\usepackage{amsmath}
\usepackage{amsfonts}
\usepackage{amssymb}
\usepackage[dvipsnames]{xcolor}
\usepackage{graphicx}
\graphicspath{{Images/}}
\usepackage[left=2cm,right=2cm,top=2cm,bottom=2cm]{geometry}
\usepackage{calc}
\usepackage{wrapfig}
\usepackage{setspace}
\usepackage{indentfirst}
\usepackage{subfigure}
\usepackage{multirow}
\usepackage{amsfonts}
\usepackage{hyperref}
\hypersetup{
    pdfstartview=FitH,  
    linkcolor=black,
    urlcolor=red, 
    colorlinks=true,
    citecolor=blue}
\usepackage{tikz}
\usetikzlibrary{ decorations.markings}

\title{Семинар №1}
\date{\today}

\author{Варламов Антоний Михайлович}

\begin{document}

	\maketitle
	\tableofcontents

	\section{Численное дифференцирование}
	
		Численное дифференцирование может потребоваться при:
		\begin{enumerate}
			\item Нет аналитической функции (пример -- экспериментальные данные
			\item Функция сложная
			\item Решение дифференциальных уравнений
			\begin{equation}
				u'_{t} = F\left(u\right);
			\end{equation}
			
			\begin{equation}
				\frac{u\left(t + \Delta t\right) - u\left(t\right)}{\Delta t} = F\left(u\left(t\right)\right)
			\end{equation}
			
			\begin{equation}
				f'\left(x\right) = \lim\limits_{h \rightarrow 0}\frac{f\left(x + h\right) - f\left(x\right)}{h} \\approx \frac{f\left(x + h\right) - f\left(x\right)}{h}
			\end{equation}
			
			Направленная разность (разностная схема)
			
			\begin{equation}
				\frac{f\left(x + h\right) - f\left(x - h\right)}{2h}	
			\end{equation}						
			Центральная разностная схема
			
			Анализ схемы:
			
			\begin{equation}
				\frac{f\left(x + h\right) - f\left(x\right)}{h} = \frac{1}{h}\left(f\left(x\right) + f'\left(x\right)\cdot h + \frac{h^2}{2!}\cdot f''\left(x\right) - f\left(x\right)\right) = f'\left(x\right) + \frac{h}{2}\cdot f''\left(\Theta\right)
			\end{equation}
			
			\begin{equation}
				\left|f'\left(x\right) - \frac{f\left(x + h\right) - f\left(x\right)}{h}\right| = \frac{h}{2}\cdot\left|f''\left(\Theta\right)\right| \leq \frac{h}{2}M_{2}
			\end{equation}
			
			\begin{equation}
				M_{2} = max\left|f''\left(x\right)\right|
			\end{equation}
			
			Порядок сходимости:
			
			\begin{equation}
				err \leq C\cdot h^{p}
			\end{equation}
			
			p - порядок сходимости
			
			Разностная схема -- схема первого порядка
			
			Анализ схемы центральных разностей
			
			\begin{eqnarray}
				\frac{f\left(x + h\right) - f\left(x - h\right)}{2h} = \\
				\frac{1}{2h}\left(f\left(x\right) + h\cdot f'\left(x\right) + \frac{h^2}{2}\cdot f''\left(x\right) + \frac{h^3}{6}\cdot f'''\left(\Theta_{1}\right)\right) - \\
				\frac{1}{2h}\left(\left(f\left(x\right) - h\cdot f'\left(x\right) + \frac{h^2}{2}\cdot f''\left(x\right) - \frac{h^3}{6}\cdot f'''\left(\Theta_{2}\right) \right)\right) = \\
				f'\left(x\right) - \frac{h^2}{12}\left(f'''\left(\Theta_{1}\right) + f'''\left(\Theta_{2}\right)\right)
			\end{eqnarray}
			
			\begin{equation}
				\left|f'\left(x\right) - \frac{f\left(x + h\right) - f\left(x - h\right)}{2h}\right| = \frac{h^2}{12}\left|f'''\left(\Theta_{1}\right) + f'''\left(\Theta_{2}\right)\right| \leq \frac{h^2}{6}\cdot M_{3}
			\end{equation}
			
			\begin{eqnarray}
				\frac{f'\left(x + h\right) - f'\left(x - h\right)}{2h} = \\ 
				\frac{1}{4h^2}\left(f\left(x + 2h\right) - f\left(x\right) - f\left(x\right) + f\left(x + 2h\right)\right)\\
				 = \frac{1}{4h^2}\left(f\left(x + 2h\right) -  2f\left(x\right) + f\left(x + 2h\right)\right)
			\end{eqnarray}
			
		\end{enumerate}
		
		Пусть имеется отрезок $\left[a; b\right]$. Наложим на отрезок сетку с шагом $h = \frac{b - a}{N}$. В таком случае:
		
		\begin{equation}
			x_{i} = a + h\cdot i, i \in \left[0, N\right]
		\end{equation}
		
		\begin{equation}
			x_{i + 1} = x_{i} + h
		\end{equation}
		
		\begin{equation}
			x_{i - 1} = x_{i} - h
		\end{equation}
		
		Будем считать:
		
		\begin{equation}
		f\left(x_{i}\right) \equiv f_{i}
		\end{equation}
		
		\begin{equation}
			f'\left(x_{i}\right) = \frac{f_{i + 1} - f_{i - 1}}{2h}
		\end{equation}
		
		Для граничных точек можно использовать:
		
		\begin{equation}
			\frac{f_{i + 1} - f_{i}}{h}
		\end{equation}
		
		
		\section{Метод неопределенных коэффициентов}
		
		Рассмотрим на примере. Определим значение $f''_{i}$
		
		Пусть схема будет иметь вид:
		
		\begin{eqnarray}
			f''_{i} = \alpha f_{i + 1} + \beta f_{i} + \gamma f_{i - 1} + \delta \ldots = \\
			\alpha\left(f_{i} + h\cdot f'_{i} + \frac{h^2}{2}f''_{i} + \frac{h^3}{6}f'''_{i} + \frac{h^4}{24}f^{IV}\left(\Theta\right)\right) + \beta \cdot f_{i} + \gamma \left(f_{i} - h\cdot f'_{i} + \frac{h^2}{2}f''_{i} - \frac{h^3}{6}f'''_{i} + \frac{h^4}{24}f^{IV}\left(\Theta\right)\right)= \\
			\left(\alpha + \beta + \gamma\right) + h\cdot f'_{i}\left(\alpha - \gamma\right) + \frac{h^2}{2}\cdot\left(\alpha + \gamma\right) + \frac{h^3}{6}\cdot \left(\alpha - \gamma\right) + \frac{h^4}{24}\cdot f^{IV}\left(\alpha + \gamma\right) 	
		\end{eqnarray}
			
		
		\begin{equation}
			\begin{cases}
				\alpha + \beta + \gamma = 0
				\\
				\alpha - \gamma = 0
				\\
				\frac{h^2}{2}\left(\alpha + \gamma\right) = 1
			\end{cases}
		\end{equation}
		
		Решением данной системы являются:
		
		\begin{equation}
			\alpha = \gamma = \frac{1}{h^2}
		\end{equation}
		
		\begin{equation}
			\beta = - \frac{2}{h^2}
		\end{equation}
		
		В таком случае:
		
		\begin{equation}
			f''_{i} \approx \frac{f_{i + 1} - 2f_{i} + f_{i - 1}}{h^2}
		\end{equation}
		
		\begin{equation}
			\frac{h^4}{24}\cdot f^{IV}\left(\alpha + \gamma\right) \approx ch^2M_{4}
		\end{equation}
\end{document}
