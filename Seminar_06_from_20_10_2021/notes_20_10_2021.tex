\documentclass[10pt,a4paper]{article}
\usepackage[utf8]{inputenc}
\usepackage[russian]{babel}
\usepackage[OT1]{fontenc}
\usepackage{amsmath}
\usepackage{amsfonts}
\usepackage{amssymb}
\usepackage[dvipsnames]{xcolor}
\usepackage{graphicx}
\graphicspath{{Images/}}
\usepackage[left=2cm,right=2cm,top=2cm,bottom=2cm]{geometry}
\usepackage{calc}
\usepackage{wrapfig}
\usepackage{setspace}
\usepackage{indentfirst}
\usepackage{subfigure}
\usepackage{multirow}
\usepackage{amsfonts}
\usepackage{hyperref}
\hypersetup{
    pdfstartview=FitH,  
    linkcolor=black,
    urlcolor=red, 
    colorlinks=true,
    citecolor=blue}
\usepackage{tikz}
\usetikzlibrary{ decorations.markings}

\title{Семинар 6}
\date{\today}

\author{Варламов Антоний Михайлович}

\begin{document}
	\maketitle
	\section{Поиск корней уравнений}
	
	Рассмотрим задачу $f\left(x\right) = 0, x \in \left[a, b\right]$
	
	Данная задача решается точно для:
	
	\begin{enumerate}
		\item Некоторые элементарные функции
		\item $P^{i}, i = \lbrace 1, 2, 3, 4\rbrace$
	\end{enumerate}
	
	Возникает необходимость в приближенных алгоритмах. Будем рассматривать 
	итерационные алгоритмы:
	
	\begin{equation}
		x_{k} \xrightarrow{k \to \infty} x
	\end{equation}
	
	Хочется найти хотя бы один корень, значит нужно определить так называемую 
	\textit{область локализации корня}.
	
	\paragraph{1. Метод деления пополам}
	
	Пусть $x^{*}$ -- корень, $x^{*} \in \left[a, b\right],  \ \ f\left(a\right)
	f\left(b\right) < 0, \ \ f \in C_{\left[a, b\right]}$
	
	Пусть $a_{0} = a, b_{0} = b$. $c_{0} = \frac{a_{0} + b_{0}}{2}$.
	
	Рассмотрим выражение $f\left(a_{0}\right)f\left(c_{0}\right)$.
	
	\begin{equation}
		f\left(a_{0}\right)f\left(c_{0}\right) < \rightarrow a_{1} = a_{0}, 
		b_{1} = c_{0}, c_{1} = \frac{a_{1} + b_{1}}{2}
	\end{equation}
	
	\begin{equation}
		f\left(c_{0}\right)f\left(b_{0}\right) < \rightarrow a_{1} = c_{0}, 
		b_{1} = b_{0}, c_{1} = \frac{a_{1} + b_{1}}{2}
	\end{equation}
	
	На итерации 0: $\left|x - x^{*}\right| \leqslant \frac{\left|a_{0} - 
	b_{0}\right|}{2}$.
	
	На итерации 1: $\left|x - x^{*}\right| \leqslant \frac{\left|a_{1} - 
	b_{1}\right|}{2} = \frac{\left|a_{0} - 	b_{0}\right|}{2^{2}}$.
	
	На итерации k: $\left|x - x^{*}\right| \leqslant  \frac{\left|a_{0} - 	
	b_{0}\right|}{2^{k}}$
	
	Критерии остановки:
	
	\begin{enumerate}
		\item $\left|f\left(x_{k}\right)\right| < \varepsilon$
		\item $\frac{\left|x_{k} - x_{k - 1}\right|}{\left|x_{k}\right|} < 
		\varepsilon$
	\end{enumerate}
	
	\paragraph{2. Метод сжимающих отображений}
	
	Введем некоторые определения:\\
	
	$\xi$ -- \textit{неподвижная точка} некоторого отображения 
	$g\left(x\right)$, если $g\left(\xi\right) = \xi$
	
	Пусть $f\left(x\right) = 0 \Rightarrow f\left(x\right) \equiv x - 
	g\left(x\right).$
	
	Значит исходная задача свелась к задаче поиска стационарных точек 
	отображения: $\xi^{k + 1} = g\left(\xi^{k}\right)$
	
	Если $\xi^{k}\xrightarrow \xi \Rightarrow x^{*} = \xi$\\
	
	$g\left(x\right)$ -- \textit{сжимающее} при $x\in \left[a, b\right]$, если:
	
	\begin{equation}
	\begin{cases}
		\forall \ x \in \left[a, b\right] g\left(x\right) \in 
		\left[a, \right]
		\\
		\left|g\left(x\right) -g\left(y\right)\right|\leqslant q\left|x-y\right|
		\forall \ x, y \in \left[a, b\right], q < 1
	\end{cases}
	\end{equation}
	
	Теорема: \textit{Если $g\left(x\right)$ -- сжимающее отображение на 
	$\left[a, b\right]$, то 
	\begin{enumerate}
		\item $\exists ! \xi \in \left[a, b\right]: \xi = g\left(\xi\right)$
		\item $\xi^{k + 1} = g\left(\xi^{k}\right), \ \ \xi^{k}\xrightarrow{
		k \to \infty} \xi$
	\end{enumerate}}
	
	Достаточные условия сжимающего отображения:
	
	\begin{equation}
		\begin{cases}
			g\left(x\right) \in \left[a, b\right] \ \forall \ x \in 
			\left[a, b\right]
			\\
			\forall \ x \in \left[a, b\right] \left|g'\left(x\right)\right| < 1
		\end{cases}
	\end{equation}
	
	Рассмотрим пример:
	
	$$x - \ln\left(x + 2\right) = 0$$
	 
	Найдем область локализации корня:
	\begin{equation}
		x = 2: 2 - \ln 4 > 0
	\end{equation}
	\begin{equation}
		x = 1: 1 - \ln 3 < 0
	\end{equation}
	
	Значит $x^{*} \in \left[1, 2\right]$
	
	Построим итерационный процесс:
	
	\begin{equation}
		x^{k} = \ln\left(x^{k} + 2\right) = g\left(x\right)
	\end{equation}
	
	Выполняются достаточные условия:
	
	\begin{equation}
		\left|\frac{1}{x + 2}\right| \leqslant \frac{1}{3} < 1, \ \ \ \ln\left(
		x + 2\right)\in \left[1, 2\right]
	\end{equation}
	
	Общий алгоритм действий:
	
	\begin{enumerate}
		\item Локализуем корни
		\item $f\left(x\right) = 0 \rightarrow x = g\left(x\right)$
		\item Доказать, что отображение является сжимающим (достаточные условия 
		выполнены, либо уточняется область локализации, либо выбирается другое
		отображение)
		\item Сколько итераций до $\varepsilon$ точности ($\left|x^{k} - 
		x^{*}\right| < q^{k}\left|a - b\right| < \varepsilon$
	\end{enumerate}
	 
\end{document}