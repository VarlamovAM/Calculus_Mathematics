\documentclass[10pt,a4paper]{article}
\usepackage[utf8]{inputenc}
\usepackage[russian]{babel}
\usepackage[OT1]{fontenc}
\usepackage{amsmath}
\usepackage{amsfonts}
\usepackage{amssymb}
\usepackage[dvipsnames]{xcolor}
\usepackage{graphicx}
\graphicspath{{Images/}}
\usepackage[left=2cm,right=2cm,top=2cm,bottom=2cm]{geometry}
\usepackage{calc}
\usepackage{wrapfig}
\usepackage{setspace}
\usepackage{indentfirst}
\usepackage{subfigure}
\usepackage{multirow}
\usepackage{amsfonts}
\usepackage{hyperref}
\hypersetup{
    pdfstartview=FitH,  
    linkcolor=black,
    urlcolor=red, 
    colorlinks=true,
    citecolor=blue}
\usepackage{tikz}
\usetikzlibrary{ decorations.markings}

\title{Семинар 8}
\date{\today}



\author{Варламов Антоний Михайлович}

\begin{document}
	\maketitle
	
	\section{Интерполяция}
	
	Пусть имеется информация в виде:
	
	\begin{table}[h!]
	\centering
	\begin{tabular}{|l|l|l|l|}
	\hline
		$x_{0}$ & $x_{1}$ & $\ldots$ & $x_{n}$ \\ \hline
		$y_{0}$ & $y_{1}$ & $\ldots$ & $y_{n}$ \\ \hline
	\end{tabular}
	\caption{}
	\end{table}	
	
	Постановка задачи:
	
	Пусть имеется некоторый отрезок $\left[a, b\right]$, на котором 
	имеются узлы $\lbrace x_{i}\rbrace_{i = 0}^{n}, x_{0} = a, x_{n} = b$.
	Будем считать также $f_{i} \approx f\left(x\right)$.
	
	\textit{Интерполят} Обозначим функцией $L\left(x\right)$. Данная функция 
	должна удовлетворять интерполяционным условиям:
	
	\begin{enumerate}
		\item $L\left(x\right) = \sum\limits_{i = 0}^{n} \alpha_{i}
		\varphi_{i}\left(x\right)$. $\varphi_{i}\left(x\right)$ --
		базисные функции интерполяции. Базисные функции линейно независимы
		на отрезке $\left[a, b\right]$.
		\item $L\left(x_{0}\right) = \sum\limits_{i = 0}^{n} \alpha_{i}
		\varphi_{i}\left(x_{0}\right)$,
		$L\left(x_{k}\right) = \sum\limits_{i = 0}^{n} \alpha_{i}
		\varphi_{i}\left(x_{k}\right)$
	\end{enumerate}
	
	Интерполяционные условия -- условия на коэффициенты разложения интерполянта 
	по базисным функциям. Данные условия приводят к линейной системе уравнений.
	В матричнм виде можно записать следующим образом:
	
	\begin{equation}
		\begin{pmatrix}
			\varphi_{0}\left(x_{0}\right) && \ldots && \varphi_{n}\left(x_{0}
			\right) \\
			\varphi_{0}\left(x_{1}\right) && \ldots && \varphi_{n}\left(x_{1}
			\right) \\
			\vdots && \vdots && \vdots \\
			\varphi_{0}\left(x_{n}\right) && \ldots && \varphi_{n}\left(x_{n}
			\right) \\
		\end{pmatrix} \begin{pmatrix}
			\alpha_{0}\\
			\alpha_{1}\\
			\vdots\\
			\alpha_{n}
		\end{pmatrix} = \begin{pmatrix}
			f_{0}\\
			f_{1}\\
			\vdots\\
			f_{n}		
		\end{pmatrix}
	\end{equation}
	
	\subsection{Алгебраическая интерполяция}
	
		Если в качестве базисных функций взять полиномы: $\varphi_{i} = x^{i}$
		
		В таком случае можно построить так называемую матрицу Ванждермонда:
		
		\begin{equation}
			\begin{pmatrix}
			1 & x_{0} & x_{0}^{2} & \ldots & x_{0}^{n}\\
			1 & x_{1} & x_{1}^{2} & \ldots & x_{1}^{n}\\			
			\vdots & \vdots & \vdots & \vdots \\
			1 & x_{n} & x_{n}^{2} & \ldots & x_{n}^{n}\\
			\end{pmatrix}
			\begin{pmatrix}
				\alpha_{0}\\
				\alpha_{1}\\
				\vdots\\
				\alpha_{n}
			\end{pmatrix}
			 = 
			 \begin{pmatrix}
				f_{0}\\
				f_{1}\\
				\vdots\\
				f_{n}		
			\end{pmatrix}
		\end{equation}
		
	\subsection{Интерполяция в форме Лагранжа}
	
		Рассмотрим в качестве базисных функций функции \textit{Лагранжа:}
		
		\begin{equation}
			l_{i}\left(x\right)  = \prod \limits_{k \neq i} 
			\frac{x - x_{k}}{x_{i} - x_{k}}
		\end{equation}
		
		В таком случае $l_{i}\left(x\right)$ -- полином степени $n$
		
		Интерполянт в таком случае:
		
		\begin{equation}
			L\left(x\right) = \sum\alpha_{i}l_{i}\left(x\right)
		\end{equation}
		
		Рассмотрим функции Лагранжа на узлах сетки:
		
		\begin{equation}
			l_{i}\left(x_{k}\right) = \prod\limits_{j \neq i}
			\frac{x - x_{j}}{x_{i} - x_{j}} = 
			\begin{cases}
				k = i: \frac{\left(x_{i} - x_{0}\right)
				\left(x_{i} - x_{1}\right)\ldots \left(x_{i} - x_{n}\right)}
				{\left(x_{i} - x_{0}\right)
				\left(x_{i} - x_{1}\right)\ldots \left(x_{i} - x_{n}\right)}
				= 1
				\\
				k \neq i : \frac{x_{k} - x_{k}}{x_{i} - x_{k}}l_{i}\left(x_{k}
				\right) = 0			
			\end{cases}
		\end{equation}
		
		Откуда делаем вывод, что $l_{i}\left(x_{k}\right) = \delta_{ik}$.
		
		Интерполянт при этом:
		
		\begin{equation}
			L\left(x_{k}\right) = \sum \alpha_{i}l_{i}\left(x\right) = \alpha_{k}
			\Rightarrow \alpha_{k} = f_{k}
		\end{equation}
		
		\begin{equation}
			L\left(x\right) = \sum\limits_{i = 0}^{n}f_{i}l_{i}\left(x\right)
		\end{equation}
		
		Последнее уравнение и описывает интерполяцию в форме Лагранжа.
		
		\paragraph*{Пример 1}
		
		\begin{table}[h!]
		\centering
		\begin{tabular}{|l|l|l|l|}
		\hline
			$x$ & $1$ & $2$ & $4$ \\ \hline
			$y$ & $7$ & $8$ & $10$ \\ \hline
		\end{tabular}
		\caption{}
		\end{table}	
		
		\begin{equation}
			L\left(x\right) = 7\cdot 
			\frac{\left(x - 2\right)\left(x - 4\right)}
			{\left(1 - 2\right)\left(1 - 4\right)} + 
			8\cdot \frac{\left(x - 1\right)\left(x - 4\right)}
			{\left(2 - 1\right)\left(2 - 4\right)} +
			10\cdot \frac{\left(x - 1\right)\left(x - 2\right)}
			{\left(4 - 1\right)\left(4 - 2\right)} 
		\end{equation}
		
		\subsection{Теорема про алгебраическую интерполяцию}
		
		\begin{enumerate}
			\item Пусть $f\left(x\right) - n - 1$ раз непрерывно 
			дифференцируема на 	$\left[a, b\right]$
			\item $x_{0}, x_{1}, \ldots, x_{n}$ -- различны
			\item $L\left(x\right)$ -- алгебраический интерполяционный 
			полином
		\end{enumerate}
		
		 	Тогда $\forall \ x \in \left[a, b\right]$:
		 	
		 	\begin{equation}
		 		E\left(x\right) = f''\left(x\right) - L\left(x\right) = 
		 		\frac{f^{\left(n - 1\right)}\left(\xi\left(x\right)\right)}
		 		{\left(n - 1\right)!} \omega\left(x\right), \ \ \
		 		\omega\left(x\right) = \prod\limits_{i = 0}^{n}
		 		\left(x - x_{i}\right)
		 	\end{equation}
		 	
		 	\begin{equation}
		 		\left|E\left(x\right)\right|\leqslant \frac{M_{n - 1}}
		 		{\left(n + 1\right)!}\left|\omega\left(x\right)\right|
		 	\end{equation}
		 	
		 \paragraph*{Пример 2} $f\left(x\right) = \frac{1}{x}, 
		 x_{0} = 2, x_{1} = 4$
		 
		 \begin{equation}
		 	L\left(x\right) = \frac{1}{2}\frac{x - 4}{2 - 4} + 
		 	\frac{1}{4}\frac{x - 2}{4 - 2}, x \in \left[a, b\right]
		 \end{equation}
		 
		 \begin{equation}
		 	f'\left(x\right) = -\frac{1}{x^{2}}, f''\left(x\right) = 
		 	\frac{1}{2x^{3}}
		 \end{equation}
		 
		 \begin{equation}
		 	M_{2} = 2\frac{1}{x_{0}^{3}} = \frac{1}{4}
		 \end{equation}
		 
		 \begin{equation}
		 	\max\limits_{x \in \left[a, b\right]} \left|\omega\left(x
		 	\right)\right| = \max\limits_{x \in \left[2, 4\right]} 
		 	\left|\left(x - 2\right)\left(x - 4\right)\right|
		 \end{equation}
		 
		 \begin{equation}
		 	x^{*} = \frac{x_{0} + x_{1}}{2}, \omega\left(x^{*}\right) = 
			\frac{\left(x_{1} - x_{0}\right)^{2}}{4} = 1		 
		 \end{equation}
		 
		 \begin{equation}
		 	E\leqslant \frac{M_{n + 1}}{\left(n + 1\right)!}\max 
		 	\left|\omega\left(x\right)\right| = \frac{1}{8}
		 \end{equation}
	
\end{document}