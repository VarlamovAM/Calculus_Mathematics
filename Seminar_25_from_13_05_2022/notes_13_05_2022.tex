\documentclass[10pt,a4paper]{article}
\usepackage[utf8]{inputenc}
\usepackage[russian]{babel}
\usepackage[OT1]{fontenc}
\usepackage{amsmath}
\usepackage{amsfonts}
\usepackage{amssymb}
\usepackage[dvipsnames]{xcolor}
\usepackage{graphicx}
\graphicspath{{Images/}}
\usepackage[left=2cm,right=2cm,top=2cm,bottom=2cm]{geometry}
\usepackage{calc}
\usepackage{wrapfig}
\usepackage{setspace}
\usepackage{indentfirst}
\usepackage{subfigure}
\usepackage{multirow}
\usepackage{amsfonts}
\usepackage{hyperref}
\hypersetup{
    pdfstartview=FitH,  
    linkcolor=black,
    urlcolor=red, 
    colorlinks=true,
    citecolor=blue}
\usepackage{tikz}
\usetikzlibrary{ decorations.markings}

\title{Семинар 25. Консультация к ПКР}
\date{\today}

\author{Варламов Антоний Михайлович}

\begin{document}

	\maketitle
	
	\section{Задача 5}
	
	\begin{equation}
		u'_{t} - cu'_{x} = 0, c > 0
	\end{equation}
	
	\begin{enumerate}
		\item Смотрим на уравнение. Каноническое уравнение переноса: $u'_{t} + 
		cu'_{x} = 0, c > 0$
		
		\item Выпускаем характеристику из верхнего узла. Судя по виду уравнения,
		характеристика <<движется>> слева направо.
		
		\begin{equation}
			u_{m}^{n + 1} = u^{*}
		\end{equation}
		
		\item Строим интерполяционный полином $\mathbb{L}\left(x\right)$ и
		находим значение в точке $x_{m} + c\Delta t$
		
	\begin{center}
\begin{tabular}{|l|l|l|l|}
\hline
$x_{m} - 2\Delta x$ & $x_{m} - \Delta x$ & $x_{m} $    & $x_{m} + \Delta x$ \\ \hline
$f^{n}_{m - 2}$     & $f^{n}_{m - 1}$    & $f^{n}_{m}$ & $f^{n}_{m - 2}$    \\ \hline
\end{tabular} 
\end{center}

		Для простоты можно вычесть $x_{m}$ и разделить все на $\Delta x$.

		\begin{center}
\begin{tabular}{|l|l|l|l|}
\hline
$-2$ & $-1$ & $0 $    & $1$ \\ \hline
$f^{n}_{m - 2}$     & $f^{n}_{m - 1}$    & $f^{n}_{m}$ & $f^{n}_{m - 2}$    \\ \hline
\end{tabular} 
\end{center}		
		
		После чего вычислить значение в точке $\sigma = \frac{c\Delta t}
		{\Delta x}$
		
		\item Для устойчивости проверить, не выходит ли характеристика за пределы шаблона.
	\end{enumerate}
	
	\section{Задача 2}
	
		\begin{enumerate}
			\item Раскладываем схему в Ряд Тейлора. Находим все члены разложения
			без множителей $\Delta x$ и $\Delta t$. Данные члены показывают, 
			какую дифференциальную задачу аппроксимирует схема.
			\item Находим члены с наименьшей скоростью убывания по $\Delta t$ и 
			$\Delta x$. Откуда получаем порядок аппроксимации. Важно проверить 
			задачу на условную аппроксимацию (отношение $\frac{\Delta x^{n}}
			{\Delta t^{m}}$
			\item Для исследования на устойчивость находим шаблон разностной 
			схемы, а дальше проверяем, чтобы характеристика не выходила за 
			пределы шаблона.
		\end{enumerate}
		
	\section{Задача 1}
		\begin{equation}
			\frac{y^{n+1}_{m} - y^{n}_{m}}{\Delta t} - a
			\frac{y^{n}_{m + 1} - 2y^{n + 1}_{m} - y^{n}_{m - 1}}{\Delta x^{2}}
			= 0
		\end{equation}
		
		\begin{enumerate}
			\item Исследование на аппроксимацию. Раскладываем схему в ряд 
			Тейлора. Смотрим проекцию точного решения на сетку и рассматриваем 
			остаточные члены.
			
			Для данной схемы:
			
			\begin{equation}
				\frac{\Delta t}{2}u''_{tt} - \frac{a\Delta x^{2}}{12}u^{IV}_{x}
				+ \frac{\Delta t}{\Delta x^{2}}\cdot 2a\cdot u'_{t}
			\end{equation}
			
			Порядок -- $O\left(\Delta t, \Delta x^{2}, \frac{\Delta t}
			{\Delta x^{2}}\right)$
			
			\item Исследование на устойчивость.
			
			Пользуемся признаком Неймана:
			
			\begin{equation}
				\frac{\lambda - 1}{\Delta t} - \frac{a}{\Delta x^{2}}\cdot
				\left(2\cos\alpha - 2\lambda\right) = 0
			\end{equation}
			
			\begin{equation}
				\lambda - 1 - \frac{a\Delta t}{\Delta x^{2}}\cdot 2\left(
				\cos\alpha - 1\right) = 0
			\end{equation}
			
			\begin{equation}
				\left|\lambda\right| = 
				\left|\frac{1 + 2\sigma^{2}\cos\alpha}{1 + 2\sigma^{2}}\right|
				\leqslant \frac{1 + 2\sigma^{2}}{1 + 2\sigma^{2}} = 1
			\end{equation}
		\end{enumerate}
		\section{Задача 4}
		
		\begin{equation}
			u'_{t} = f\left(t, u\right)
		\end{equation}
		
		\begin{enumerate}
			\item
			Общий вид схем Адамса:
			
			\begin{equation}
				\frac{y^{n + 1}_{m} - y^{n}_{m}}{\Delta t} = \alpha_{0}f^{n + 1}
				+ \alpha_{1}f^{n} + \alpha_{2}f^{n - 1} + \cdot + \alpha_{k}
				f^{n + 1 - k}
			\end{equation}
			
			\item 3 порядок $\rightarrow$ 3 коэффициента
			\item Так как схема явная -- $\alpha_{0} = 0$.
			\item Раскладываем схему в ряд Тейлора, зануляем коэффициенты перед
			множителями со степенями $\Delta t$.
			\item После зануления коэффициентов получаем систему уравнений, 
			решая которую находим коэффициенты.
			\item Общий вид формул дифференцирования назад:
		
			Явные:
				\begin{equation}
					\alpha_{1}y^{n + 1} + \alpha_{2}y^{n} + \alpha_{3}y^{n - 1} 
					+ \ldots = f^{n}
				\end{equation}
				
			Неявные:
				
				\begin{equation}
					\alpha_{1}y^{n + 1} + \alpha_{2}y^{n} + \alpha_{3}y^{n - 1} 
					+ \ldots = f^{n + 1}
				\end{equation}
		\end{enumerate}
		\section{Задача 3}
		Рекомендуется посмотреть в задачнике номер 14.8.9 или конспект 24 
		семинара.
		\section{Задача 5}
		Рекомендуется посмотреть в задачнике номер 12.8.9
		\section{Задача 6}
		
		\begin{equation}
			\Delta u = f
		\end{equation}
		
		Данная задача сводится к задаче обращения матрицы. В задачнике на стр.
		264 приведена формула:
		
		\begin{equation}
			N_{iter} = \gamma\cdot\left|\ln\varepsilon\right|
		\end{equation}
		
		Пару замечаний о методе простых итераций:
		
		\begin{equation}
			x^{n + 1} = x^{n} + \tau\left(b - Ax^{n}\right)
		\end{equation}
		
		\begin{equation}
			\tau_{\text{опт}} = \frac{2}{\lambda_{\max} + \lambda_{\min}}
		\end{equation}
		
		\begin{equation}
			q_{\text{опт}} = \frac{\lambda_{\max} - \lambda_{\min}}
			{\lambda_{\max} + \lambda_{\min}}
		\end{equation}
		
		Тогда норма ошибки:
		
		\begin{equation}
			\parallel \delta^{n}\parallel = q\cdot\parallel\delta^{n - 1}
			\parallel = q^{n}\parallel\delta^{0}\parallel
		\end{equation}
		
		\begin{equation}
			\frac{\parallel\delta^{n}\parallel}{\parallel\delta^{0}\parallel}
			=q^{n}\leqslant \varepsilon
		\end{equation}
		
		\begin{equation}
			n\cdot\left|\ln q\right| \geqslant  \left|\ln\varepsilon\right|
		\end{equation}
\end{document}