\documentclass[10pt,a4paper]{article}
\usepackage[utf8]{inputenc}
\usepackage[russian]{babel}
\usepackage[OT1]{fontenc}
\usepackage{amsmath}
\usepackage{amsfonts}
\usepackage{amssymb}
\usepackage[dvipsnames]{xcolor}
\usepackage{graphicx}
\graphicspath{{Images/}}
\usepackage[left=2cm,right=2cm,top=2cm,bottom=2cm]{geometry}
\usepackage{calc}
\usepackage{wrapfig}
\usepackage{setspace}
\usepackage{indentfirst}
\usepackage{subfigure}
\usepackage{multirow}
\usepackage{amsfonts}
\usepackage{hyperref}
\hypersetup{
    pdfstartview=FitH,  
    linkcolor=black,
    urlcolor=red, 
    colorlinks=true,
    citecolor=blue}
\usepackage{tikz}
\usetikzlibrary{ decorations.markings}

\title{семинар 23}
\date{\today}

\author{Варламов Антоний Михайлович}

\begin{document}
	\maketitle
	
	\section{Решение уравнений мелкой воды}
	
	Рассмотрим гиперболическую систему уравнений мелкой воды:
	
	\begin{equation}
		\begin{cases}
			\frac{\partial u}{\partial t} = -g\frac{\partial h}{\partial x}
			\\
			\frac{\partial h}{\partial t} = -H\frac{\partial u}{\partial x}
		\end{cases}
	\end{equation}
	
	Запишем схему аппроксимации для такой системы:
	
	\begin{equation}
		\begin{cases}
			\frac{u^{n + 1}_{m} - u^{n}_{m}}{\Delta t} =
			 -g\frac{h^{n}_{m + 1} - h^{n}_{m - 1}}{2\Delta x}
			\\
			\frac{h^{n + 1}_{m} - h^{n}_{m}}{\Delta t} =
			 -H\frac{u^{n}_{m + 1} - u^{n}_{m - 1}}{2\Delta x}
		\end{cases}
	\end{equation}
	
	Охарактеризуем данную схему:
	
	\begin{enumerate}
		\item Аппроксимация. Порядок аппроксимации данной схемы $O\left(
		\Delta t, \Delta x^{2}\right)$
		\item Устойчивость
		\begin{equation}
			\begin{pmatrix}
				u \\ h
			\end{pmatrix}^{n}_{m} = \lambda^{n}\cdot
			\begin{pmatrix}
				\hat{u} \\ \hat{h}
			\end{pmatrix}\cdot e^{-\alpha m}
		\end{equation}
		
		Получаем условие:
		
		\begin{equation}
			\begin{cases}
				\frac{\lambda - 1}{\Delta t}\cdot\hat{u} = -gi\frac{\sin\alpha}
				{\Delta x}\hat{h}
				\\
				\frac{\lambda - 1}{\Delta t}\cdot\hat{h} = -Hi\frac{\sin\alpha}
				{\Delta x}\hat{u}
			\end{cases}
		\end{equation}
		
		Получаем матрицу:
		
		\begin{equation}
			\hat{A} = \begin{pmatrix}
				\frac{\lambda - 1}{\Delta t} & ig\frac{\sin\alpha}{\Delta x}
				\\
				iH\frac{\sin\alpha}{\Delta x} & \frac{\lambda - 1}{\Delta t}
			\end{pmatrix}
		\end{equation}
		
		\begin{equation}
			\det A = \left(\frac{\lambda - 1}{\Delta t}\right)^{2} + gH
			\frac{\sin^{2}\alpha}{\Delta x^{2}} = 0
		\end{equation}
		
		Используя данное условие, выражаем $\lambda$ и рассматриваем вопрос 
		устойчивости. Конкретно для данной схемы приходим к выводу 
		неустойчивости схемы.
	\end{enumerate}
	
	\section{Решение волнового уравнения}
	
	Рассмотрим уравнение
	
	\begin{equation}
		\frac{\partial^{2} \psi}{\partial t^{2}} = k^{2}\frac{\partial^{2} \psi}
		{\partial x^{2}}
	\end{equation}
	
	Для данного уравнения используем схему:
	
	\begin{equation}
		\frac{\psi_{m}^{n + 1} - 2\psi_{m}^{n} + \psi_{m}^{n - 1}}{\Delta t^{2}}
		= k^{2}\frac{\psi_{m + 1}^{n} - 2\psi_{m}^{n} + \psi_{m - 1}^{n}}
		{\Delta x^{2}}
	\end{equation}
	
	Повторим исследование:
	
	\begin{enumerate}
		\item Аппроксимация. Схема имеет порядок аппроксимации $O\left(
		\Delta t^{2}, \Delta x^{2}\right)$
		\item Устойчивость
		\begin{equation}
			\frac{\lambda^{2} - 2\lambda + 1}{\Delta t^{2}} = 
			\lambda\frac{k^{2}}{\Delta x^{2}}\left( - 4\sin^2\frac{\alpha}{2}
			\right)
		\end{equation}
		
		\begin{equation}
			\lambda^{2} - 2\lambda + 1 = \lambda\frac{k^{2}\Delta t^{2}}
			{\Delta x^{2}}\left(-4\sin^{2}\frac{\alpha}{2}\right)
		\end{equation}
		
		\begin{equation}
			\lambda^{2} + \lambda\left(-2 + 4\sigma\sin^{2}\frac{\alpha}{2}
			\right) + 1 = 0
		\end{equation}
		
		Хотим
		\begin{equation}
			\begin{cases}
				\left|\lambda\right| \leqslant 1, \ \text{корни не кратные}
				\\
				\left|\lambda\right|  < 1, \ \text{корни кратные}
			\end{cases}
		\end{equation}
		
		Из теоремы Виета:
		
		\begin{equation}
			\lambda_{1}\cdot\lambda_{2} = 1 \Rightarrow \lambda_{1} = 
			\overline{\lambda_{2}}
		\end{equation}
		
		\begin{equation}
			D = \left(-2 + 4\sigma\sin^{2}\frac{\alpha}{2}\right)^{2} - 4 < 0
		\end{equation}
		
		\begin{equation}
			\left(-4 + 4\sigma\sin^{2}\frac{\alpha}{2}\right)\cdot
			4\sigma\sin^{2}\frac{\alpha}{2} < 0
		\end{equation}
		
		\begin{equation}
			\sigma^{2}\sin^{2}\frac{\alpha}{2} < 1\Rightarrow \sigma < 1
		\end{equation}
		
		Это означает, что:
		
		\begin{equation}
			\frac{k\Delta t}{\Delta x} < 1
		\end{equation}
		
		Для данного уравнения можно дополнительно провести исследование с 
		помощью характеристик.
	\end{enumerate}
	
	\subsection{Задание начальных условий для волнового уравнения}
	
	Для корректной постановки задачи необходимо:
	
	\begin{equation}
		\psi\left(0, x\right) = \alpha\left(x\right)\mapsto\psi_{m}^{0}
		\\
		\psi'_{t}\left(0, x\right) = \beta\left(x\right)
	\end{equation}
	
	Представим 
	
	\begin{equation}
		\psi^{1}_{m} = \psi_{m}^{0} + \Delta t\left(\psi'_{t}\right)_{m}^{0}
	\end{equation}
	
	В таком случае:
	
	\begin{equation}
		\psi^{1}_{m} = \alpha_{m} + \Delta t\cdot\beta_{m}
	\end{equation}
	
	Если разложить функцию дальше, получим:
	
	\begin{equation}
		\psi^{1}_{m} = \psi_{m}^{0} + \Delta t\left(\psi'_{t}\right)_{m}^{0} + 
		\frac{\Delta t^{2}}{2}\left(\psi''_{tt}\right)_{m}^{0} = 
		\psi_{m}^{0} + \Delta t\left(\psi'_{t}\right)_{m}^{0} + 
		\frac{\Delta t^{2}}{2}k^{2}\cdot\left(\psi''_{xx}\right)
	\end{equation}
	
	В таком случае:
	
	\begin{equation}
		\psi^{1}_{m} = \alpha_{m} + \Delta t\cdot\beta_{m} + \frac{\Delta t^{2}}
		{2}\cdot k^{2}\cdot\frac{\alpha_{m + 1} - 2\alpha_{m} + \alpha_{m - 1}}
		{\Delta x^{2}}
	\end{equation}
	
\end{document}