\documentclass[10pt,a4paper]{article}
\usepackage[utf8]{inputenc}
\usepackage[russian]{babel}
\usepackage[OT1]{fontenc}
\usepackage{amsmath}
\usepackage{amsfonts}
\usepackage{amssymb}
\usepackage[dvipsnames]{xcolor}
\usepackage{graphicx}
\graphicspath{{Images/}}
\usepackage[left=2cm,right=2cm,top=2cm,bottom=2cm]{geometry}
\usepackage{calc}
\usepackage{wrapfig}
\usepackage{setspace}
\usepackage{indentfirst}
\usepackage{subfigure}
\usepackage{multirow}
\usepackage{amsfonts}
\usepackage{hyperref}
\hypersetup{
    pdfstartview=FitH,  
    linkcolor=black,
    urlcolor=red, 
    colorlinks=true,
    citecolor=blue}
\usepackage{tikz}
\usetikzlibrary{ decorations.markings}

\title{Семинар 24}
\date{\today}

\author{Варламов Антоний Михайлович}

\begin{document}
	\maketitle
	
	Рассмотрим систему уравнений:
	
	\begin{equation}
		\frac{\partial \vec{V}}{\partial t} + \hat{A}\frac{\partial \vec{V}}
		{\partial x} = \vec{f}
	\end{equation}
	
	В качестве примера возьмем систему:
	
	\begin{equation}
		\begin{cases}
			\frac{\partial u}{\partial t} + 2\frac{\partial u}{\partial x} - 
			4\cdot\frac{\partial v}{\partial x} = f\left(x, t\right)
			\\
			\frac{\partial v}{\partial t} - 3\frac{\partial u}{\partial x} + 
			3\cdot\frac{\partial v}{\partial x} = g\left(x, t\right)
		\end{cases}
	\end{equation}
	
	Первое, что стоит сделать -- выполнить проверку на принадлежность системы 
	к гиперболическим системам.
	
	Для этого необходимо определить спектр матрицы $\hat{A}$.
	
	\begin{equation}
		\det\begin{pmatrix}
			2 - \lambda & -4 \\
			-3 & 3 - \lambda
		\end{pmatrix} = \lambda^{2} - 5\lambda - 6 = 0 \Rightarrow 
		\lambda_{1, 2} = -1, 6
	\end{equation}
	
	Система будет гиперболической, если $\lambda\left(\hat{A}\right) \in 
	\mathbb{R}$.
	
	Далее следует найти левые собственные векторы для матрицы $\hat{A}$:
	
	\begin{equation}
		\xi^{T}\hat{A} = \lambda\xi^{T}
	\end{equation}
	
	В таком случае, после транспонирования:
	
	\begin{equation}
		\hat{A}^{T} \xi = \lambda\xi
	\end{equation}
	
	Значит, левый собственные вектора -- собственные вектора 
	\textbf{транспонированной матрицы} -- $\hat{A}^{T}$
	
	Для нашей матрицы левые собственные вектора будут следующие:
	
	\begin{equation}
		\xi_{1} = \begin{pmatrix}
			1 \\ 1
		\end{pmatrix}; 
		\xi_{2} = \begin{pmatrix}
			3 \\ -4
		\end{pmatrix}
	\end{equation}
	
	Домножим исходное векторное уравнение на левый собственный вектор:
	
	\begin{equation}
		\xi_{1}^{T}\frac{\partial \vec{V}}{\partial t} + \xi_{1}^{T}\hat{A}
		\frac{\partial \vec{V}}{\partial x} = \xi_{1}^{T}\vec{f}
	\end{equation}
	
	После преобразования, получаем:
	
	\begin{equation}
		\frac{\partial}{\partial t}\left(\xi_{1}^{T}\vec{V}\right) + 
		\lambda_{1}\frac{\partial}{\partial x}\left(\xi_{1}^{T}\vec{V}\right) =
		\xi_{1}^{T}
	\end{equation}
	
	В таком случае можно определить \textit{инвариант Римана}:
	
	\begin{equation}
		R_{1} = \xi_{1}^{T}\vec{V}, R_{2} = \xi_{2}^{T}\vec{V}
	\end{equation}
	
	В терминах инвариантов, получаем:
	
	\begin{equation}
		\begin{cases}
			\frac{\partial R_{1}}{\partial t} + \lambda_{1}\cdot
			\frac{\partial R_{1}}{\partial x} = \xi_{1}^{T}\vec{f}
			\\
			\frac{\partial R_{2}}{\partial t} + \lambda_{2}\cdot
			\frac{\partial R_{2}}{\partial x} = \xi_{2}^{T}\vec{f}
		\end{cases}
	\end{equation}
	
	Данные инварианты \textit{сохраняются } на характеристиках.
	
	Для нашего условия, получаем:
	
	\begin{equation}
		R_{1} = \xi_{1}^{T}\vec{V} = u + v
	\end{equation}
	\begin{equation}
		R_{2} = \xi_{2}^{T}\vec{V} = 3u - 4v
	\end{equation}
	
	Данный шаг сводит задачу к решению двух уравнений переноса.
	
	Далее, можно проанализировать граничные условия.
	
	Рассмотрим несколько вариантов:
	
	\begin{equation}
		u\left(x, 0\right) = \varphi_{1}; v\left(x, 0\right) = \varphi_{2}
	\end{equation}
	
	\begin{enumerate}
		\item $u\left(0, t\right) = \psi_{1}\left(t\right),
			 v\left(0, t\right) = \psi_{2}\left(t\right) $ 
		\item $u\left(0, t\right) = \psi_{1}\left(t\right),
			 v\left(1, t\right) = \psi_{2}\left(t\right) $
		\item $u\left(0, t\right) + v\left(0, t\right) = \psi_{1}\left(t\right),
			 v\left(1, t\right) = \psi_{2}\left(t\right) $
		\item $u\left(0, t\right) = \psi_{1}\left(t\right),
			 v\left(1, t\right) + u\left(1, t\right) = \psi_{2}\left(t\right) $
	\end{enumerate}
	
		
		Для первого уравнения на инварианты Римана характеристики направлены 
		<<справа налево>>. Для решения задачи, $R_{1}$ необходимо знать или 
		уметь вычислить на правой границе. Для второго уравнения характеристики 
		направлены  <<слева направо>>, а значит нужно знать или уметь вычислить 
		$R_{2}$ на левой границе. 
		
		%Нужно очень сильно доработать данный материал в итоговом коспекте!
		
		Анализом возможности выразить инварианты, приходим к выводу о 
		корректности 2 и 4 граничных условий.
		
	Дальше можно пытаться найти схемы с различными условиями:
	
	\begin{enumerate}
		\item Монотонную
		\item 2-го порядка
	\end{enumerate}

	Для монотонной схемы подойдет <<Левый или Правый уголок>> (В зависимости 
	от знака $\lambda$):
	
	\begin{equation}
		\frac{\left(R_{1}\right)^{n + 1}_{j} - \left(R_{1}\right)^{n }_{j}}
		{\Delta t} - \lambda_{1}
		\frac{\left(R_{1}\right)^{n}_{j + 1} - \left(R_{1}\right)^{n}_{j}}
		{\Delta x} = \left(f_{1}\right)_{j}^{n}
	\end{equation}
	
	Для схемы второго порядка можно использовать схему Лакса Вендрофа:
	
	\begin{equation}
		\frac{\left(R_{1}\right)^{n + 1}_{j} - \left(R_{1}\right)^{n}_{j}}
		{\Delta t} + \lambda_{1}
		\frac{\left(R_{1}\right)^{n}_{j + 1} - \left(R_{1}\right)^{n1}_{j - 1}}
		{2\Delta x} - 
		\frac{\lambda_{1}^{2}\Delta t}{2\Delta x^{2}}\left(
		\left(R_{1}\right)^{n}_{j + 1} - 2\left(R_{1}\right)^{n}_{j} + 
		\left(R_{1}\right)^{n}_{j - 1}\right) = 0
	\end{equation}
	
	При условии $\frac{\left|\lambda_{1}\right|\Delta t}{\Delta x} \leqslant 1$
	
	\section{Эллиптические уравнения}
	
	Рассмотрим уравнение Пуассона:
	
	\begin{equation}
		\begin{cases}
			\Delta \psi = f
			\\
			\psi_{\partial \Omega} = g
		\end{cases}
	\end{equation}
	
	Рассмотрим решение в <<коробке>>
	
	\begin{equation}
		\frac{\partial^{2}\psi}{\partial x^{2}} + \frac{\partial^{2}\psi}
		{\partial y^{2}} = f^{2}
	\end{equation}
	
	Формально зададим сетку:
	
	\begin{equation}
		x_{i} = i\cdot\Delta x, \Delta x= \frac{1}{N_{x}}
	\end{equation}
	\begin{equation}
		y_{i} = i\cdot\Delta y, \Delta y= \frac{1}{N_{y}}
	\end{equation}
	
	\begin{equation}
		\frac{\psi_{i + 1, j} - 2\psi_{i, j} + \psi_{i - 1, j}}{\Delta x^{2}} + 
		\frac{\psi_{i + 1, j} - 2\psi_{i, j} + \psi_{i + 1, j}}{\Delta y^{2}} = 
		f_{i, j}
	\end{equation}
	
	Данная схема имеет порядок $O\left(\Delta x^{2}, \Delta y^{2}\right)$.
	
	Данное уравнение можно записать в матричном виде:
	
	\begin{equation}
		\hat{A}\vec{\psi} = \vec{f}
	\end{equation}
	
	\begin{equation}
		\vec{\psi} = \left[\psi_{1, 1}, \psi_{2, 1}, \ldots, \psi_{N_{x} - 1}, 
		\psi_{1, 2}, \psi_{2, 2}, \ldots, \psi_{N_{x} - 1, N_{y} - 1}\right]
	\end{equation}
	
	При таком определении матрица $\hat{A}$ становится 5-ти диагональной, а 
	решение задачи сводится к обращению данной матрицы.
	
\end{document}