\documentclass[10pt,a4paper]{article}
\usepackage[utf8]{inputenc}
\usepackage[russian]{babel}
\usepackage[OT1]{fontenc}
\usepackage{amsmath}
\usepackage{amsfonts}
\usepackage{amssymb}
\usepackage[dvipsnames]{xcolor}
\usepackage{graphicx}
\graphicspath{{Images/}}
\usepackage[left=2cm,right=2cm,top=2cm,bottom=2cm]{geometry}
\usepackage{calc}
\usepackage{wrapfig}
\usepackage{setspace}
\usepackage{indentfirst}
\usepackage{subfigure}
\usepackage{multirow}
\usepackage{amsfonts}
\usepackage{hyperref}
\hypersetup{
    pdfstartview=FitH,  
    linkcolor=black,
    urlcolor=red, 
    colorlinks=true,
    citecolor=blue}
\usepackage{tikz}
\usetikzlibrary{ decorations.markings}

\title{Семинар 9}
\date{\today}

\author{Варламов Антоний Михайлович}

\begin{document}
	\maketitle
	
	\section{Интерполяция}
	
	Постановка задачи:
	
	\begin{equation}
	\centering
	\begin{tabular}{|l|l|l|l|}
	\hline
		$x_{0}$ & $x_{1}$ & $\ldots$ & $x_{n}$ \\ \hline
		$f_{0}$ & $f_{1}$ & $\ldots$ & $f_{n}$ \\ \hline
	\end{tabular} \rightarrow P_{n}\left(x\right)
	\end{equation}
	
	\begin{equation}
		P_{n}\left(x\right) = \sum\limits \alpha_{i}x^{i}
	\end{equation}
	
	Запись интерполянта в форме Лагранжа:
	
	\begin{equation}
		L_{n}\left(x\right) = \sum\limits_{i = 0}^{n}f_{i}\cdot l_{i}\left(x
		\right)
	\end{equation}
	
	где базисные функции:
	
	\begin{equation}
		l_{i}\left(x\right) = \prod\limits_{k \neq i} \frac{x - x_{i}}
		{x_{i} - x_{k}}
	\end{equation}
	
	Ошибка интерполяции оценивается как:
	
	\begin{equation}
		\left|E\left(x\right)\right| = \left|f\left(x\right) - P_{n}\left(x
		\right)\right| = \left|\frac{f^{\left(n + 1\right)}\left(\xi\right)}
		{\left(n + 1\right)!}\cdot \omega\left(x\right)\right|
	\end{equation}
	
	где $\omega\left(x\right) = \prod\limits_{k = 0}^{n}\left(x - x_{k}\right)$
	
	Приведем оценки $\left|\max\omega\left(x\right)\right|$:
	
	Для равномерной сетки:
	
	\begin{equation}
		\left|\omega\left(x\right)\right| \leqslant \frac{h^{n+1}n!}{4}, \ \ 
		h = x_{i + 1} - x_{i} = \frac{b - a}{n + 1}
	\end{equation}
	
	Как видно, ошибка интерполяции имеет двойную природу. Ошибка складывается 
	из ошибок, связанных с интерполируемой функцией и ошибок, связанных с сеткой
	, на которой происходит интерполяция. Значит, если есть возможность выбирать 
	сетку, то можно минимизировать ошибки сетки. Общая постановка задачи:
	
	\begin{equation}
		\min\limits_{x_{0}, x_{1}, \ldots, x_{i}, \ldots x_{n}}\max_{x \in 
		\left[a, b\right]}\left|\omega\left(x\right)\right|
	\end{equation}
	
	Данная задача решена, решение -- разбиение через нули полиномов Чебышева:
	
	\begin{equation}
		x_{i} = \frac{1}{2}\left(b + a - \left(b - a\right)\cdot\cos
		\left(\frac{2i + 1}{2\left(n + 1\right)}\pi\right)\right)
	\end{equation}
	
	В таком случае, для сетки Чебышева:
	
	\begin{equation}
		\left|\omega\left(x\right)\right| \leqslant \frac{2\left(b - a
		\right)^{n + 1}}{4^{n + 1}}
	\end{equation}
	
	Интерполяция сходится не всегда. Классический пример несходящегося 
	интерполяционного процесса-- функция Рунге: 
	
	\begin{equation}
		f\left(x\right) = \frac{1}{1 + \left(5x\right)^{2}}, x \in 
	\left[-1, 1\right]
	\end{equation}
	
	Рассмотрим, как влияют ошибки входных данных:
	
	\begin{equation}
		\centering
	\begin{tabular}{|l|l|l|l|}
	\hline
		$x_{0}$ & $x_{1}$ & $\ldots$ & $x_{n}$ \\ \hline
		$f_{0}$ & $f_{1}$ & $\ldots$ & $f_{n}$ \\ \hline
	\end{tabular} 
	\rightarrow 
	\centering
	\begin{tabular}{|l|l|l|l|}
	\hline
		$x_{0}$ & $x_{1}$ & $\ldots$ & $x_{n}$ \\ \hline
		$f_{0}$ & $f_{1}$ & $\ldots$ & $f_{n}$ \\ \hline
		$f_{0} + \delta f_{0}$ & $f_{1}+ \delta f_{1}$ & 
		$\ldots$ & $f_{n} + \delta f_{0}$ \\ \hline
	\end{tabular} 
	\end{equation}
	
	При условии:
	
	\begin{equation}
		\left|\delta f_{i}\right| \leqslant \varepsilon \ \forall \ i
	\end{equation}
	
	В таком случае можно построить два интерполянта:
	
	\begin{equation}
		P_{n}\left(x, f\right) = \sum\limits_{i = 0}^{n}l_{i}\left(x\right)\cdot
		f_{i}
	\end{equation}
		
	\begin{equation}
		P_{n}\left(x, f + \delta f\right) = \sum\limits_{i = 0}^{n}l_{i}\left(x
		\right)\cdot\left(f_{i} + \delta f\right)
	\end{equation}
	
	\begin{equation}
		\left|P_{n}\left(x, f + \delta f\right) - P_{n}\left(x, f\right)\right| 
		\leqslant \varepsilon \left|\sum l_{i}\left(x\right)\right| \leqslant
		\varepsilon \sum\left|l_{i}\left(x\right)\right| \leqslant
		\varepsilon \max\limits_{x} \sum\left|l_{i}\left(x\right)\right|
	\end{equation}
	
	Где $\sum\left|l_{i}\left(x\right)\right| = \lambda_{n}\left(x\right)$ --
	 функция Лебега, а 
	$\max\limits_{x} \sum\left|l_{i}\left(x\right)\right| = \Lambda_{n}$ --
	 константа Лебега.
	
	Приведем некоторые оценки:
	
	\paragraph{Uniform grid}
	
	\begin{equation}
		\Lambda_{n} \leqslant \frac{2^{n + 3}}{n}
	\end{equation}
	
	\paragraph{Cheb grid}
	
	\begin{equation}
		\Lambda_{n} \leqslant \frac{2}{\pi}\log\left(n + 1\right) + 1
	\end{equation}
	
	\begin{equation}
		\left|E\right| \leqslant E^{method} + e^{roundcing}
	\end{equation}
	
	\subsection{Интерполяция в форме Ньютона}
	
		Построим интерполянт в виде:
		
	\begin{equation}
		P_{n}\left(x\right) = a_{0} + a_{1}\left(x - x_{0}\right) + 
		a_{2}\left(x - x_{0}\right)\left(x - x_{1}\right) + \ldots + 
		a_{n}\left(x - x_{0}\right)\left(x - x_{1}\right)\ldots\left(x - x_{n}
		\right) = \sum\limits_{i = 0}^{n}a_{i}\prod\limits_{j = 0}^{i}\left(
		x - x_{j}\right)
	\end{equation}
	
	Для определения коэффициентов необходимо записать интерполяционные условия.
	
	Если матрицу, полученную при записи интерполяционных условий диагонализовать 
	, то метод сведется к методу Лагранжа.
	
	\subsection{Разделенные разности}
	
	Введем так называемые разделенные разности различных порядков:
	
	\begin{equation}
		f\left[x_{0}\right]\rightleftharpoons f_{0}
	\end{equation}
	
	\begin{equation}
		f\left[x_{0}, x_{1}\right] \rightleftharpoons \frac{f_{1} - f_{0}}
		{x_{1} - x_{0}} = \frac{f\left[x_{1}\right] - f\left[x_{0}\right]}{
		x_{1} - x_{0}}
	\end{equation}
	
	\begin{equation}
		f\left[x_{0},\ldots , x_{n}\right] \rightleftharpoons 
		\frac{f\left[x_{1},\ldots, x_{n}\right] - 
		f\left[x_{0}, \ldots, x_{n - 1}\right] \rightleftharpoons }{x_{n} - 
		x_{0}}
	\end{equation}
	
	В таком случае:
	
	\begin{equation}
		P_{n}\left(x\right) = f\left[x_{0}\right] + f\left[x_{0}, x_{1}\right] 
		\left(x - x_{0}\right) + f\left[x_{0}, x_{1}, x_{2}\right]
		\left(x - x_{0}\right)\left(x - x_{1}\right)\left(x - x_{2}\right) 
	\end{equation}
	
	\begin{equation}
		P_{n}\left(x\right) = a_{0} + \left(x - x_{0}\right)\cdot \left(a_{1} + 
		\left(x - x_{1}\right)\left(a_{2} + \ldots \right)\ldots\right)
	\end{equation}
	
	\subsection{Метод обратной интерполяции}
	
	\begin{equation}]
		\centering
	\begin{tabular}{|l|l|l|l|}
	\hline
		$x_{0}$ & $x_{1}$ & $\ldots$ & $x_{n}$ \\ \hline
		$f_{0}$ & $f_{1}$ & $\ldots$ & $f_{n}$ \\ \hline
	\end{tabular} 
	\rightarrow 
	\centering
	\begin{tabular}{|l|l|l|l|}
	\hline
		$f_{0}$ & $f_{1}$ & $\ldots$ & $f_{n}$ \\ \hline
		$x_{0}$ & $x_{1}$ & $\ldots$ & $x_{n}$ \\ \hline
	\end{tabular} 
	\end{equation}	
	
	\begin{equation}
		x^{*} \rightarrow x\left(0\right)\simeq p_{n}\left(0\right)
	\end{equation}
	
	\paragraph{Пример}
	
	\begin{equation}
		\sin\left(x\right) - \exp\left(x\right) = 0
	\end{equation}
	
	для следующих данных:
	
	\begin{equation}
		\begin{tabular}{|l|l|l|l|l|}
	\hline
		$x_{i}$ & -4 & -3.5 & -3 & -2.5 \\ \hline
		$f_{i}$ & 0.738 & 0.321 & -0.191 & -0.681 \\ \hline
	\end{tabular} 
	\end{equation}
	
	Необходимо построить обратный интерполянт и найти $P\left(0\right)$
	
\end{document}