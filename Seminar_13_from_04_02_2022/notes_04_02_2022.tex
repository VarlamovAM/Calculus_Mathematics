\documentclass[10pt,a4paper]{article}
\usepackage[utf8]{inputenc}
\usepackage[russian]{babel}
\usepackage[OT1]{fontenc}
\usepackage{amsmath}
\usepackage{amsfonts}
\usepackage{amssymb}
\usepackage[dvipsnames]{xcolor}
\usepackage{graphicx}
\graphicspath{{Images/}}
\usepackage[left=2cm,right=2cm,top=2cm,bottom=2cm]{geometry}
\usepackage{calc}
\usepackage{wrapfig}
\usepackage{setspace}
\usepackage{indentfirst}
\usepackage{subfigure}
\usepackage{multirow}
\usepackage{physics}
\usepackage{amsfonts}
\usepackage{hyperref}
\hypersetup{
    pdfstartview=FitH,  
    linkcolor=black,
    urlcolor=red, 
    colorlinks=true,
    citecolor=blue}
\usepackage{tikz}
\usetikzlibrary{ decorations.markings}

\title{Семинар 13}
\date{\today}

\author{Варламов Антоний Михайлович}

\begin{document}
	\maketitle
	
	\tableofcontents
	
	\vspace{1cm}
	
	Данный семинар посвящен понятию разностных уравнений. В первой части
	рассматриваются \textit{линейные разностные стационарные уравнения}, а во 
	второй \textit{системы линейных разностных стационарных уравнений}.	
	
	\section{Линейные разностные стационарные уравнения}
	
	\subsection{Терминология}
	
	\textit{\textbf{Определение}} Линейным разностным уравнением будем называть
	уравнение вида:
	
	\begin{equation}
		y_{k + n} + a_{1}\cdot y_{k + n - 1} + \ldots + a_{n}\cdot y_{k} = f_{k}
	\end{equation}
	
	где $a_{i} \ \forall \ i \in \mathbb{R}, a_{n}\neq 0, k \in \mathbb{N_{0}}$.
	
	Для начала рассмотрим \textit{однородное} уравнение: разностное уравнение, у 
	которого нулевая правая часть ($f_{k} \equiv 0$):
	
	\subsection{Однородные разностные уравнения. Понятие характеристического 
	многочлена уравнения и оператора трансляции. Корни характеристического 
	многочлена и их кратность}
	
	\begin{equation}
		y_{k + n} + a_{1}\cdot y_{k + n - 1} + \ldots + a_{n}\cdot y_{k} = 0
	\end{equation}
	
	Разберемся, в каком виде стоит пытаться искать решения такого уравнения. 
	Введем \textit{оператор сдвига:}
	
	\begin{equation}
		T y_{k} = y_{k + 1}
	\end{equation}
	
	Рассмотрим важное свойство такого оператора:
	
	\begin{equation}
		T^{n} y_{k} = T^{n - 1}T y_{k} = T^{n - 1}y_{k + 1} = T^{n - 2}Ty_{k + 
		1} = \ldots = T y_{k + n - 1} = y_{k + n}
	\end{equation}
	
	Используя данный оператор, можно перейти к следующему виду разностного 
	однородного уравнения:
	
	\begin{equation}
		\left(T^{n} + a_{1}T^{n - 1} + \ldots + a_{n}\right)y_{k} = 0
	\end{equation}
	
	Рассмотрим данное уравнение в виде:
	
	\begin{equation}
		\hat{L}(T)y_{k} = 0
	\end{equation}
	
	Рассматривая $\hat{L}\left(T\right)$ как многочлен относительно $T$, 
	выполним \textit{приведение} данного многочлена. В таком случае, получим:
	
	\begin{equation}
		\hat{L}\left(T\right) = \prod\limits_{i = 1}^{k}
		\left(T - \lambda_{i}\right)^{\alpha_{i}}
	\end{equation}
	
	Рассмотрим "простейший элемент" разложения $\left(T - \lambda_{i}\right)^{
	\alpha_{i}}$ и определим его действие на $y_{k}$: 
	
	Пусть $\alpha_{i} = 1$. Тогда 
	\begin{equation}
		\left(T - \lambda_{i}\right)y_{k} = y_{k + 1} - \lambda_{i}y_{k} = 0
	\end{equation}
	
	\begin{equation}
		\Rightarrow y_{k} = \lambda_{i}^{k}
	\end{equation}
	
	При $\alpha_{i} > 1$:
	
	\begin{equation}
		\left(T - \lambda_{i}\right)^{\alpha_{i}}y_{k}		
	\end{equation}
	
	Как и в теории дифференциальных уравнений, воспользуемся следующим методом:
	Рассмотрим решения в виде $y_{k} = \lambda^{k}\varphi_{k}$. Оказывается, что 
	в таком случае, используя свойства характеристического многочлена 
	относительно кратности корня ($L^{\left(i\right)}\left(\lambda_{0}\right) = 
	0 \ \ \forall \ i \leqslant s - 1$, где $s$ -- кратность корня $\lambda_{0}$
	) решениями (линейно независимыми) являются функции:
	
	\begin{equation}
		\begin{cases}
			\varphi_{k, 0} = \lambda^{k};
			\\
			\varphi_{k, 1} = \lambda^{k}\cdot k;
			\\
			\varphi_{k, 2} = \lambda^{k}\cdot k^{2};
			\\
			\vdots
			\\
			\varphi_{k, s - 1} = \lambda^{k}\cdot k^{s - 1};
		\end{cases}
	\end{equation}
	
	Таким образом, рассматривая возможные решения линейного разностного 
	уравнения приходим к понятию характеристического многочлена:
	
	\begin{equation}
		\chi\left(\lambda\right) = 
		\lambda^{n} + a_{1}\lambda^{n - 1} + \ldots + a_{n} = 0
	\end{equation}
	
	Важное замечание относительно корней характеристического многочлена. Так как
	по определению разностного уравнения $n$-ого порядка $a_{n} \neq 0$, то и 
	$\lambda \neq 0$.
	
	Для линейного стационарного разностного уравнения справедлив принцип 
	линейности (суперпозиции) решения (Аналогично свойству линейного 
	дифференциального уравнения с постоянными коэффициентами). Данные принцип 
	выражается записывается следующим образом:\\
	
	\textit{Если $y^{k}_{I}, y^{k}_{II}, \ldots$ -- решения уравнения, то 
	$\widetilde{y} = C_{1}y^{k}_{I} + C_{2}y^{k}_{II} + \ldots$ -- также 
	является решением.}\\
	
	Рассматривая все корни характеристического многочлена (с учетом их 
	кратности), можно прийти к следующим выводам:
	
	В том случае, когда все корни $\lambda_{1}, \lambda_{2}, \ldots, \lambda_{n}
	\in \mathbb{R} $ и  $\ \forall \ i, j \lambda_{i} \neq \lambda_{j}$, решения
	вида $\lambda_{i}^{k}$ образуют ФСР.
	
	Для доказательства этого факта рассмотрим определитель следующего вида:
	
	\begin{equation}
		\begin{vmatrix}
			\lambda_{1}^{k} & \lambda_{2}^{k} & \ldots & \lambda_{n}^{k} \\
			\lambda_{1}^{k + 1} & \lambda_{2}^{k + 1} & \ldots & 
			\lambda_{n}^{k + 1} \\
			\vdots & \vdots & \ddots & \vdots \\
			\lambda_{1}^{k + n - 1} & \lambda_{2}^{k + n - 1} & \ldots &
			\lambda_{n}^{k + n - 1}
		\end{vmatrix}
	\end{equation}
	
	Воспользовавшись свойствами определителя, нетрудно перейти к виду:
	
	\begin{equation}
	\begin{aligned}
		\begin{vmatrix}
			\lambda_{1}^{k} & \lambda_{2}^{k} & \ldots & \lambda_{n}^{k} \\
			\lambda_{1}^{k + 1} & \lambda_{2}^{k + 1} & \ldots & 
			\lambda_{n}^{k + 1} \\
			\vdots & \vdots & \ddots & \vdots \\
			\lambda_{1}^{k + n - 1} & \lambda_{2}^{k + n - 1} & \ldots &
			\lambda_{n}^{k + n - 1}
		\end{vmatrix} &=&
		\lambda_{1}^{k}\cdot
		\begin{vmatrix}
			1 & \lambda_{2}^{k} & \ldots & \lambda_{n}^{k} \\
			\lambda_{1} & \lambda_{2}^{k + 1} & \ldots & 
			\lambda_{n}^{k + 1} \\
			\vdots & \vdots & \ddots & \vdots \\
			\lambda_{1}^{n - 1} & \lambda_{2}^{k + n - 1} & \ldots &
			\lambda_{n}^{k + n - 1}
		\end{vmatrix} &=& \\ 		
		\lambda_{1}^{k}\lambda_{2}^{k}\cdot
		\begin{vmatrix}
			1 & 1 & \ldots & \lambda_{n}^{k} \\
			\lambda_{1} & \lambda_{2} & \ldots & 
			\lambda_{n}^{k + 1} \\
			\vdots & \vdots & \ddots & \vdots \\
			\lambda_{1}^{n - 1} & \lambda_{2}^{n - 1} & \ldots &
			\lambda_{n}^{k + n - 1}
		\end{vmatrix} &=& \prod\limits_{i = 1}^{n}\lambda_{i}^{k}
		\begin{vmatrix}
			1 & 1 & \ldots & 1 \\
			\lambda_{1} & \lambda_{2} & \ldots & 
			\lambda_{n} \\
			\vdots & \vdots & \ddots & \vdots \\
			\lambda_{1}^{n - 1} & \lambda_{2}^{n - 1} & \ldots &
			\lambda_{n}^{n - 1}
		\end{vmatrix} &=&
		\\ \prod\limits_{i = 1}^{n}\lambda_{i}^{k}\cdot W\left(
		\lambda_{1}, \lambda_{2}, \ldots, \lambda_{n}\right)&\neq 0
	\end{aligned}
	\end{equation}
	
	Так как определитель Вандермонда отличен от 0 для попарно различных чисел.
	
	Рассмотрим пример, иллюстрирующий алгоритм нахождения решения разностного
	уравнения в том случае, когда все корни имеют кратность равную 1:
	
	\textit{\textbf{Пример 1.}} \begin{equation}
		y_{k + 2} + y_{k + 1} - 2y_{k} = 0
	\end{equation}
	
	Характеристический многочлен $\chi\left(\lambda\right) = \lambda^{2} + 
	\lambda - 2$
	
	Находим корни характеристического уравнения:
	
	\begin{equation}
		\chi\left(\lambda\right) = 0 \Rightarrow 
		\begin{cases}
			\lambda_{1} = -2
			\\
			\lambda_{2} = 1	
		\end{cases}
	\end{equation}
	
	Таким образом ФСР данного уравнения будет представлена двумя решениями:
	
	\begin{equation}
		\begin{cases}
			y_{k}^{1} = \left(-2\right)^{k}
			\\
			y_{k}^{2} = 1^{k} \equiv 1
		\end{cases}
	\end{equation}
	
	Значит общее решение уравнения:
	
	\begin{equation}
		\widetilde{y}_{k} = C_{1}\cdot\left(-2\right)^{k} + C_{2}
	\end{equation}\\
	
	Рассмотрим ФСР разностного уравнения в том случае, когда существуют корни 
	кратности больше 1. Пусть $\lambda_{1}, \lambda_{2}, \ldots, \lambda_{s}, 
	\left(s \in \mathbb{N}, 1 \leqslant s < n\right)$ -- корни 
	характеристического многочлена $\chi\left(\lambda\right)$ исходного 
	уравнения, а $m_{1}, m_{2}, \ldots, m_{s}, \left(\sum\limits_{i = 1}^{s}
	m_{i} = n\right)$ -- кратности данных корней.
	
	Так как для "простейшего элемента" разложения характеристического многочлена 
	решением является ФСР следующего вида:
	
	\begin{equation}
		\begin{cases}
			\varphi_{k, 0} = \lambda^{k};
			\\
			\varphi_{k, 1} = \lambda^{k}\cdot k;
			\\
			\varphi_{k, 2} = \lambda^{k}\cdot k^{2};
			\\
			\vdots
			\\
			\varphi_{k, s - 1} = \lambda^{k}\cdot k^{s - 1};
		\end{cases}
	\end{equation}
	
	то общее решение такого уравнения будет выглядеть следующим образом:
	
	\begin{equation}
		\widetilde{y}_{k} = \sum\limits_{i = 0}^{s}\sum\limits_{j = 0}^
		{m_{i} - 1} \left(C_{ij}\cdot k^{j}\cdot\lambda_{i}^{k}\right)
	\end{equation}
	
	\textit{\textbf{Пример 2.}} \begin{equation}
		y_{k + 2} - 4y_{k + 1} + 4y_{k} = 0
	\end{equation}
	
	Характеристический многочлен:  $\chi\left(\lambda\right) = 
	\left(\lambda - 2\right)^{2}$
	
	У данного многочлена единственный корень $\lambda_{1}= 2$ кратности $m = 2$.
	
	В таком случае, общее решение представляется в виде:
	
	\begin{equation}
		\widetilde{y}_{k} = \left(C_{1} + C_{2}\cdot k\right)\left(2\right)^{k}
	\end{equation}
	
	\subsection{Комплекснозначные решения разностного уравнения. Приведение 
	к действительному виду}
	
	Немного вернемся к представлению характеристического многочлена, а 
	конкретно, скажем пару слов о итоговом виде, в котором мы его записали. 
	Как известно, разложение многочлена на простые линейные множители возможно в 
	поле комплексных чисел, но не действительных. Значит, необходимо 
	дополнительно рассматривать комплекснозначные значения корней 	
	характеристического многочлена. Для данного рассмотрения используем две 
	Леммы:
	
	\textit{\textbf{Лемма 1}}: \textit{Функция $y_{k} = u_{k} + iv_{k}$ 
	является решением стационарного разностного однородного уравнения тогда и 
	только тогда, когда $u_{k} = \Re y_{k}, v_{k} = \Im y_{k}$ -- решения этого
	уравнения.}
	
	\textit{\textbf{Лемма 2}}: \textit{Если $\lambda_{0} = \alpha + i\beta$ -- 
	комплексный корень кратности $m$ характеристического многочлена, то и 
	$\vec{\lambda_{0}} = \alpha - i\beta$ также является корнем той же 
	кратности.}
	
	Используя данные Леммы нетрудно перейти к новому виду записи решения 
	уравнения, характеристический многочлен которого имеет комплекснозначный 
	корень кратности $m$.
	
	\begin{equation}
		\begin{cases}
			u_{lk} = k^{l}\lambda^{k}\cos k\varphi
			\\
			v_{lk} = k^{l}\lambda^{k}\sin k\varphi
		\end{cases}
	\end{equation}
	
	Где $l \in \left[0, m - 1\right] $. Рассмотрим конкретный пример:
	
	\textit{\textbf{Пример 3.}}
	
	\begin{equation}
		y_{k + 2} + y_{k} = 0
	\end{equation}
	
	Переходя к характеристическому многочлену, получаем:
	
	\begin{equation}
		\lambda^{2} + 1 = 0 \Rightarrow \lambda = \pm i
	\end{equation}
	
	В таком случае, удобно представить полученные корни в виде:
	
	\begin{equation}
		\lambda = \exp\left(\frac{i\pi}{2}\right) = \cos\left(
		\frac{\pi}{2}\right) + i\sin\left(\frac{\pi}{2}\right);
	\end{equation}
	
	\begin{equation}
		\lambda^{k} = \exp\left(i\frac{k\pi}{2}\right) = 
		\cos\left(\frac{k\pi}{2}\right) + i\sin\left(\frac{k\pi}{2}\right)
	\end{equation}
	
	Выделяя действительную часть полученных выражений, получаем, что:
	
	\begin{equation}
		\widetilde{y}_{k} = С_{1}\cos\left(\frac{k\pi}{2}\right) + 
		C_{2}\sin\left(\frac{k\pi}{2}\right)
	\end{equation}
	
	Итак, на этом обсуждение однородных уравнений заканчивается. Дальше 
	возникает естественный вопрос, а что делать если исходное уравнение было 
	неоднородным? Как измениться алгоритм рассуждений в зависимости от вида 
	правой части уравнения? Ответим на эти вопросы, рассматривая 
	\textit{неоднородные уравнения}
	
	\subsection{Неоднородные разностные уравнения. Метод Вариации постоянной.
	Построение общего решения неоднородного уравнения с использованием частного
	решения неоднородного уравнения и общего решения однородного. 
	Квазимногочлены в правой части неоднородного разностного уравнения}
	
	
\end{document}