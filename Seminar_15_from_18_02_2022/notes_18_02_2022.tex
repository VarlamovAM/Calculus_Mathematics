\documentclass[10pt,a4paper]{article}
\usepackage[utf8]{inputenc}
\usepackage[russian]{babel}
\usepackage[OT1]{fontenc}
\usepackage{amsmath}
\usepackage{amsfonts}
\usepackage{amssymb}
\usepackage[dvipsnames]{xcolor}
\usepackage{graphicx}
\graphicspath{{Images/}}
\usepackage[left=2cm,right=2cm,top=2cm,bottom=2cm]{geometry}
\usepackage{calc}
\usepackage{wrapfig}
\usepackage{setspace}
\usepackage{indentfirst}
\usepackage{subfigure}
\usepackage{multirow}
\usepackage{amsfonts}
\usepackage{hyperref}
\hypersetup{
    pdfstartview=FitH,  
    linkcolor=black,
    urlcolor=red, 
    colorlinks=true,
    citecolor=blue}
\usepackage{tikz}
\usetikzlibrary{ decorations.markings}

\title{Семинар 15}
\date{\today}

\author{Варламов Антоний Михайлович}

\begin{document}
	\maketitle
	
	\section{Исследование разностных схем на аппроксимацию}
	
	Вернемся к задаче вида:
	
	\begin{equation}\label{eq:problem}
		\begin{cases}
			\frac{d{u}}{dt} = {f}\left(u, t\right), \ t\in[0, T]
			\\
			{u}\left(0\right) = {u}_{0}
		\end{cases}
	\end{equation}
	
	Для приближенного решения данной задачи можно, например, воспользоваться 
	явным методом Эйлера:
	
	\begin{equation}\label{eq:explcit_Euler}
		\begin{cases}
			\frac{y_{i + 1} - y_{i}}{\Delta t} = {f}\left(y_{i}, t_{i}\right), \ 
			i=1..N_T-1
			\\
			y_{0} = u_{0}
		\end{cases}.
	\end{equation}
	
	Введем в рассмотрение понятие аппроксимации разностной схемой исходного 
	дифференциального уравнения.
	Для этого перепишем уравнения (\ref{eq:problem}) и разностную схему 
	(\ref{eq:explcit_Euler}) в операторном виде.
	
	Для уравнения (\ref{eq:problem}) операторная запись -- $\mathbb{L}\left(u
	\right)=\mathbb{F}$, где
	
	\begin{equation}
		\mathbb{L}\left(u\right)  = 
		\begin{cases}
			\frac{du}{dt} - f\left(t, u\right), \ \ &t > 0
			\\
			u\left(0\right) , \ \ &t = 0
		\end{cases},
		\
		F  = 
		\begin{cases}
		0, \ \ &t > 0
		\\
		u_0 , \ \ &t = 0
		\end{cases}.
	\end{equation}
	
	Для разностной схемы (\ref{eq:explcit_Euler}) операторная запись -- $
	\mathbb{L}_h y =\mathbb{F}_h$, где
	
	\begin{equation}
		\mathbb{L}_{h} = 
		\begin{cases}
			\frac{y_{i} - y_{i-1}}{\Delta t} - {f}\left(y_{i-1}, t_{i-1}
			\right), \ &i \in \left[1, N_T\right]
			\\
			y_{0}, \  &i = 0
		\end{cases},
		\mathbb{F}_{h} = 
		\begin{cases}
		0, &i \in \left[1, N_T\right]
		\\
		u_{0}, &i = 0
		\end{cases}.
	\end{equation}
	
	В данном случае явная схема Эйлера рассматривается в качестве примера, 
	аналогичным образом можно ввести операторную запись
	для любой другой схемы. 
	
	Будем говорить, что разностная схема $\mathbb{L}_h y =\mathbb{F}_h$ 
	аппроксимирует исходное уравнение $\mathbb{L}\left(u\right)=\mathbb{F}$,
	если выполняется
	
	\begin{equation}
	\parallel r \parallel \equiv 
	\parallel \mathbb{L}_{h}\mathbb{P}_{h}u - \mathbb{F}_{h}\parallel 
	\xrightarrow{\Delta t \to 0} 0.
	\end{equation}
	
	Здесь $\mathbb{P}_{h}u$ -- проекция точного решения на сетку, которая может 
	быть определена, например, как $\left(\mathbb{P}_{h}u\right)_i = u(t_i)$.
	Если удается показать, что 
	
	\begin{equation}
	\parallel \mathbb{L}_{h}\mathbb{P}_{р}u - \mathbb{F}_{h}\parallel 
	\leqslant C\cdot \Delta t^{p},
	\end{equation}
	то будем говорить, что разностная схема имеет порядок аппроксимации $p$.
	
	Исследуем порядок аппроксимации для явной схемы Эйлера. Рассмотрим вектор 
	невязки (компоненты проекции точного решения на сетку будем для краткости 
	обозначать как $u_i$):
	
	\begin{equation}
		r_{i} \equiv \frac{u_{i + 1} - u_{i}}{\Delta t} - f\left(t_{i}, u_{i}
		\right)
	\end{equation}
	
	При этом:
	
	\begin{equation}
		u_{i + 1} = u\left(t_{i + 1}\right) = u\left(t_{i} + \Delta t\right) = 
		u\left(t_{i}\right) + u'\left(t_{i}\right)\cdot \Delta t + 
		\frac{u''\left(\theta_{i}\right)}{2}\Delta t^{2}, \ \theta_{i} \in 
		\left[
		t_{i}, t_{i + 1}\right].
	\end{equation}
	
	Значит
	
	\begin{align}
		r_{i} = \frac{1}{\Delta t}\left(u_{i} + u'_{i}\Delta t + 
		\frac{u''\left(\theta_{i}\right)}{2}\Delta t^2 - u_{i}\right) - 
		f\left(t_{i}, u_{i}\right) 	= \frac{\Delta t}{2} \cdot 
		u''\left(\theta_{i}\right).
	\end{align}
	
	В последнем равенстве мы воспользовались тем, что $u_i$ -- значение точного 
	решения в точке $t_i$, а значит оно удовлетворяет в этой точке исходному 
	уравнению, то есть $u'_i=f(t_i, u_i)$.
	
		\begin{align}
	&\parallel r_{i} \parallel_\infty = \max_i |r_i| = \max_i \left| 
	\frac{\Delta t}{2} \cdot 
	u''\left(\theta_{i}\right)\right|\leqslant\frac{M_2}{2}\Delta t, \\
	&M_2 = \max_{t\in[0,T]} |u''(t)|. 
	\end{align}
	
	В качестве еще одного примера рассмотрим следующую схему и исследуем ее на 
	аппроксимацию
	
	\begin{equation}
		\frac{y_{i + 1} - y_{i}}{\Delta t} = \frac{1}{2}\left(f\left(t_{i + 1}, 
		y_{i + 1}\right) + f\left(t_{i}, y_{i}\right)\right)
	\end{equation}
	
	Данная схема называется неявный метод трапеций или схема Кранка-Николсон.
	
	\begin{equation}
		r_{i} = \frac{u_{i + 1} - u_{i}}{\Delta t} - \frac{1}{2}\left(
		f\left(t_{i + 1}, 
		y_{i + 1}\right) + f\left(t_{i}, y_{i}\right)\right)
	\end{equation}
	
	Для упрощения выкладок воспользуемся свойством точного решения $u'(t_{i
	+1})=f(t_{i+1}, u_{i+1})$, $u'(t_{i})=f(t_{i}, u_{i})$
	
	\begin{align}
		r_{i} = \frac{u_{i + 1} - u_{i}}{\Delta t} - \frac{1}{2}\left(u'_{i + 1}
		+ u'_{i}\right) = \frac{1}{\Delta t}\cdot \left(u_{i} + \Delta t\cdot
		u'_{i} + \frac{\Delta t^2}{2}u''_{i} + \frac{\Delta t^3}{6}u'''_{i}
		\left(\theta_{i}\right)- u_{i}\right)  -
		\\
		-\frac{1}{2}\left(u'_{i} + u''_{i}\Delta t + \frac{u'''\left(\xi_{i}
		\right)}{2} + u'_{i}\right) = 
		\frac{\Delta t}{2}u''_{i} + \frac{\Delta t^2}{6}u'''\left(\theta_{i}
		\right) - \frac{u''_{i}\Delta t}{2} - \frac{u'''\left(\xi_{i}\right)}{4}
		\Delta t^{2} = 
		\\
		=\Delta t^{2}\left(\frac{u'''\left(\theta_{i}\right)}{6} - 
		\frac{u'''\left(\xi_{i}\right)}{4}\right).
	\end{align}
	
	Отсюда видно, что схема имеет второй порядок аппроксимации.
	
	\subsection{Явные методы Рунге-Кутты}
	
		Данные методы являются многостадийными, то есть для выполнения одного 
		шага по времени (переход от $y_n$ к $y_{n+1}$), необходимо провести 
		дополнительные вспомогательные вычисления.
		
		Пример двухстадийной схемы Рунге-Кутты второго порядка:
		
		\begin{equation}\label{eq:rk_example}
			\frac{\widetilde{y} - y_{i}}{\frac{\Delta t}{2}} = f\left(t_{i}, 
			y_{i}\right)\ \ \mapsto \ \ \frac{y_{i + 1} - y_{i}}{\Delta t} = 
			f\left(t_{i + \frac{1}{2}}, \widetilde{y}\right)
		\end{equation}
		
		Перейдем к рассмотрению общего вида явных методов Р.-К. Для это введем 
		вспомогательные величины:
		
		\begin{equation}
			T_{i} = t_{n} + c_{i}\Delta t,
		\end{equation}
		
		\begin{equation}
			Y_{i} = y_{n} + \Delta t \cdot \sum\limits_{k = 1}^{i - 1}f\left(
			T_{k}, Y_{k}\right)\cdot \beta_{ik}.
		\end{equation}
		
		Для явных методов договоримся, что первая стадия всегда тривиальная:
		
		\begin{equation}
		T_{1} \equiv t_{n},
		\end{equation}
		
		\begin{equation}
		Y_{1} \equiv y_{n}.
		\end{equation}
		
		Вычислив вспомогательные величины, можем сделать шаг по времени:
		
		\begin{equation}
			y_{n + 1} = y_{n} + \Delta t\cdot \sum\limits_{k = 1}^{N_{k}}
			b_{k}\cdot f\left(T_{k}, Y_{k}\right)
		\end{equation}
		
		$N_{k}$ -- количество стадий.
		
		Существует зависимость между максимальным порядком аппроксимации и 
		количеством стадий:
		
		\begin{equation}
			\begin{matrix}
				2 \to 2
				\\
				3 \to 3
				\\
				4 \to 4
				\\
				\hspace{4.cm} 5 \to 4 \ \text{Первый барьер Бутчера}
				\\
				6 \to 5
				\\
				\vdots
			\end{matrix}
		\end{equation}
		
		Для компактной записи методов Р.-К. используют так называемые 
		\textit{таблицы Бутчера}. Таблица Бутчера для общего вида:
		
		\begin{table}[h!]
\centering
\begin{tabular}{lllllll}
\multicolumn{1}{l|}{$0$}  & 0            &              &              &          
&             &         \\
\multicolumn{1}{l|}{$c_{2}$}  & $\beta_{21}$ & 0            &              &          
&             &         \\
\multicolumn{1}{l|}{$c_{3}$}  & $\beta_{31}$ & $\beta_{32}$ & 0            &          
&             &         \\
\multicolumn{1}{l|}{$\vdots$} & $\vdots$     & $\vdots$     & $\ddots$     & 0        
&             &         \\
\multicolumn{1}{l|}{$c_{N_k}$}  & $\beta_{N_k1}$ & $\beta_{N_k2}$ & $
\beta_{N_k3}$ & $
\ldots$ & $\beta_{N_k N_{k-1}}$           &     0    \\ \hline
                              & $b_{1}$      & $b_{2}$      & $b_{3}$      & $
                              \ldots$ & $b_{N_k - 1}$ & $b_{N_k}$
\end{tabular}
\caption{Таблица Бутчера для общего вида явного метода Р.К. с числом стадий $N_k
$.}
\label{tab:Butcher_tab_gen_view}
\end{table}

Перепишем разностную схему (\ref{eq:rk_example}) с использованием новых 
обозначений:
		
		\begin{equation}
			\begin{cases}
				Y_{1} = y_{n}, T_{1} = t_{n}, \ \text{(тривиаильная первая 
				стадия)}
				\\
				Y_{2} = y_{n} + \frac{\Delta t}{2}\cdot f\left(T_{1}, Y_{1}
				\right), \ T_2 = t_n+\frac{\Delta t}{2}
				\\
				y_{n + 1} = y_{n} + \Delta t\left(0 \cdot f\left(T_{1}, Y_{1}
				\right) + f\left(T_{2}, Y_{2}\right)\right)
			\end{cases}
		\end{equation}
		
Таблица Бутчера для данного метода:		
		
		\begin{table}[h!]
\centering
\begin{tabular}{l|ll}
\multicolumn{1}{l|}{0}             & 0             & 0 \\
\multicolumn{1}{l|}{$\frac{1}{2}$} & $\frac{1}{2}$ & 0 \\ \hline
                                   & 0             & 1
\end{tabular}
\caption{Таблица Бутчера для описанного выше метода}
\label{tab:Butcher_tab_example_1}
\end{table}

		Напоследок, выпишем таблицу классического метода Рунге-Кутты:
		
		\begin{table}[h!]
\centering
\begin{tabular}{l|llll}
\multicolumn{1}{l|}{0}             & 0             &               &               
&               \\
\multicolumn{1}{l|}{$\frac{1}{2}$} & $\frac{1}{2}$ & 0             &               
&               \\
\multicolumn{1}{l|}{$\frac{1}{2}$} & 0             & $\frac{1}{2}$ & 0             
&               \\
\multicolumn{1}{l|}{1}             & 0             & 0             & 1             
& 0             \\ \hline
                                   & $\frac{1}{6}$ & $\frac{1}{3}$ & $\frac{1}
                                   {3}$ & $\frac{1}{6}$
\end{tabular}
\caption{Таблица Бутчера классического метода Рунге-Кутты 4 порядка}
\label{tab:classic_Runge_Cutt}
\end{table}

		Условия на коэффициенты методов Р.-К. для достижения заданного порядка 
		аппроксимации:
		
		\begin{equation}
			\begin{cases}
				\sum\limits_{k}\beta_{ik} = c_{i} \ \ &\text{0 порядок, 
				необязательные условия Рунге}
				\\
				\sum b_{i} = 1&\text{1 порядок}
				\\
				\sum b_{i}c_{i} = \frac{1}{2}&\text{2 порядок}
				\\
				\begin{cases}
					\sum b_{i}c_{i}^{2} = \frac{1}{3}
					\\
					\sum\limits_{i}\sum\limits_{j}b_{j}\beta_{ji}c_{i} = 
					\frac{1}{6}
				\end{cases}&\text{3 порядок}
			\end{cases}
		\end{equation}
		
		
\end{document}